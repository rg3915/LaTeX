\documentclass[a4paper]{report}
\usepackage[utf8]{inputenc}
\usepackage[T1]{fontenc}
\usepackage[brazil]{babel}
\usepackage[normalem]{ulem}
\usepackage{amssymb,amsfonts,amsmath,indentfirst}
\usepackage{enumerate,graphicx,etoolbox}
\usepackage{textcomp,marvosym,wasysym,eurosym,mathrsfs,pifont}
\usepackage{lipsum,exscale,fix-cm}
\usepackage{anttor}
\usepackage[T1]{pbsi} %brushscr
\usepackage{calligra}
% \usepackage{concrete}
% \usepackage[clock,weather]{ifsym}
% \usepackage[math]{iwona}
\usepackage{yfonts}
\usepackage[margin=3cm]{geometry}
\usepackage{microtype}
\usepackage{hyperref}
\hypersetup{pdfpagelayout=SinglePage, %TwoPageLeft
  bookmarksopen=true,
  colorlinks=true,
  urlcolor=blue,
  linkcolor=black,
  pdftitle={Tipografia},
  pdfauthor={R\'egis S. Santos}
}

%\DeclareMathAlphabet\mathpazo{OML}{zplm}{m}{it}

% Modo matem\'atico em tamanho normal
% \everymath{\displaystyle}
% Define o caminho das figuras.
\graphicspath{{/media/HD/LaTeX/site/fig_site/tipografia}}

% Figuras
\newcommand{\myfig}[2][width=5cm]{
    \begin{figure}[!htb]
      \centering
      \includegraphics[#1]{/media/HD/LaTeX/site/fig_site/tipografia/#2}
    \end{figure}
}

% familia
\newcommand{\fonte}[2]{
  {\fontfamily{#1}\selectfont #2}
}

% Frase
\newcommand{\frase}{Tem Coelhos que fogem de Abelhas Brigando por Jabuti porque faz chover uva e xadrez.}

\DeclareFixedFont{\fonteUtopiaItalico}{T1}{put}{m}{it}{10}

% Fonte padrao deste documento: Computer Modern Roman
\renewcommand\familydefault{cmr}
% Fonte padrao para typewriter
\renewcommand\ttdefault{cmtt}

\makeatletter
\patchcmd{\tableofcontents}{\@starttoc{toc}}{\thispagestyle{empty}\pagestyle{empty}\@starttoc{toc}}{}{}
\makeatother

\title{Tipografia\\ Manipulando fontes no \LaTeX}
\author{\url{http://latexbr.blogspot.com}}
\date{2012}
%*******************************************************
\begin{document}

\maketitle

\tableofcontents
\clearpage
\pagestyle{plain}
\cleardoublepage

\chapter{Tipografia}

\href{http://pt.wikipedia.org/wiki/Tipografia}{Tipografia} é a essência do \href{http://latexbr.blogspot.com/2010/04/introducao-ao-latex.html}{LaTeX}. Tipografia trata essencialmente de fontes e seus estilos, então veremos aqui um estudo completo sobre o uso de fontes no LaTeX. A escolha de uma fonte requer 4 características essenciais: família (family), série (serie), forma (shape) e tamanho (size). A seguir, veremos cada uma delas.

A seguir abordaremos os seguintes tópicos sobre \emph{tipografia}:

% <ul>
% <li>\href{#subt1>Codificação de caracteres}</li>
% <li>\href{#subt2>Acentos, símbolos e caracteres especiais}</li>
% <li>\href{#subt3>Enfatizando um texto}</li>
% <li>\href{#subt4>Família (family)}</li>
% <li>\href{#subt5>Séries (series)}</li>
% <li>\href{#subt6>Formas (shape)}</li>
% <li>\href{#subt7>Tamanho (size)}</li>
% <li>\href{#subt8>Outras fontes}</li>
% <li>\href{#subt9>Fontes matemáticas}</li>
% <li>\href{#subt10>Fontes grandes}</li>
% <li>\href{#subt11>Catálogo de fontes}</li>
% </ul>

Um exemplo simples:

\verb|{\textit{\textsf{texto itálico sem serifa}}}|

\begin{center}
{\textit{\textsf{texto itálico sem serifa}}}
\end{center}

\

Uma forma geral de manipulação de fontes com o comando \verb|\selectfont| é:

\begin{verbatim}
{\fontfamily{familia}\fontseries{serie}\fontshape{forma}
\fontsize{size}{baselineskip}\selectfont texto}
\end{verbatim}

Baixe \href{}{selectfont.tex}

Porém, veremos mais adiante cada detalhe destes comandos e veremos que existem outras formas de manipular as fontes.

A fonte padrão do LaTeX é \href{http://en.wikipedia.org/wiki/Computer_Modern}{Computer Modern}, criado por \href{http://en.wikipedia.org/wiki/Donald_Knuth}{Donald Knuth}. E o tamanho padrão é 10pt.

\begin{verbatim}
\familydefault = \rmdefault = Computer Modern Roman
\fontsize{10}{12} = \normalsize
\end{verbatim}

\verb|\familydefault| significa \emph{família} padrão de fonte e \verb|\rmdefault| significa fonte no estilo \emph{romano}.

\verb|\fontsize{10}{12}| é o \emph{tamanho} da fonte (será explicado melhor mais pra frente) e \verb|\normalsize| é a fonte no seu tamanho padrão (normal).

Veja a seguir um \href{}{guia rápido} dos principais comandos para manipulação de fontes.

\

* GUIA RÁPIDO (+ comandos)

* Fazer um resumo de todas as famílias usadas aqui.

\href{http://pt.wikipedia.org/wiki/Pangrama}{Pangrama}

\

Nota: Onde aparecer \verb|texdoc| significa que você pode localizar a documentação de um pacote pelo \emph{terminal} digitando \verb|texdoc nome-do-pacote|.


\section{Codificação de caracteres}

A codificação de caracteres vai além do ASCII. Para que o LaTeX reconheça os caracteres acentuados como 'á' ou 'ç' (cê-cedilha), por exemplo, é necessário uma codificação de \emph{entrada} e uma de \emph{saída}. A codificação é importante, pois dependendo do sistema operacional ou do editor usado os caracteres (acentuados) podem não ser interpretados corretamente.

O pacote que gerencia codificação de \textbf{entrada} é o

\verb|\usepackage{inputenc}|

Para \href{http://latexbr.blogspot.com/2011/10/lancado-ubuntu-1110-oneiric-ocelot.html}{Linux} a codificação padrão é a \href{http://pt.wikipedia.org/wiki/Unicode}{unicode}, então use a opção

\verb|\usepackage[utf8]{inputenc}|

Para \emph{Windows} o padrão é ISO, então use

\verb|\usepackage[latin1]{inputenc}|

Para \emph{Mac} experimente UTF8 ou applemac

\verb|\usepackage[applemac]{inputenc}|

Caso haja a necessidade de converter seus arquivos ISO para UTF8, ou vice-versa, leia \href{http://latexbr.blogspot.com/2011/01/recodificando-seus-arquivos-iso-para.html}{Recodificando seus arquivos ISO para UTF8}.

O pacote que gerencia codificação de \textbf{saída} é o

\verb|\usepackage{fontenc}|

por padrão use a opção

\verb|\usepackage[T1]{fontenc}|

\

Para os exemplos a seguir vamos definir um preâmbulo básico:

\begin{verbatim}
\documentclass[a4paper]{article}
% cod. entrada
\usepackage[utf8]{inputenc} %Linux - ou [latin1] para Win
% cod. saida
\usepackage[T1]{fontenc}
% idioma
\usepackage[brazil]{babel}

\begin{document}
  ...
\end{document}
\end{verbatim}

\section{Acentos, símbolos e caracteres especiais}

Usando a codificação correta o LaTeX aceita entradas de acentos na forma tradicional (exemplo, \emph{é ação}), porém o LaTeX possui uma entrada especial de caracteres acentuados. Sendo assim, a frase \emph{é ação} pode ser digitada como \verb|\'e a\c c\~ao|. Este método pode ser usado sempre, independente da codificação usada. Veja a tabela a seguir:

%*\#tabela: ja existe on line
% \myfig[width=7cm]{acentos}
\begin{center}
\begin{tabular}{@{}l@{\ }ll@{\ }ll@{\ }ll@{\ }ll@{\ }l@{}}
\hline
\`o   & \verb!\`o! &
\'o   & \verb!\'o! &
\^o   & \verb!\^o! &
\~o   & \verb!\~o! &
\=o   & \verb!\=o! \\
\.o   & \verb!\.o! &
\"o   & \verb!\"o! &
\c o  & \verb!\c o! &
\v o  & \verb!\v o! &
\H o  & \verb!\H o! \\
\c c  & \verb!\c c! &
\d o  & \verb!\d o! &
\b o  & \verb!\b o! &
\t oo & \verb!\t oo! &
\oe   & \verb!\oe! \\
\OE   & \verb!\OE! &
\ae   & \verb!\ae! &
\AE   & \verb!\AE! &
\aa   & \verb!\aa! &
\AA   & \verb!\AA! \\
\o    & \verb!\o! &
\O    & \verb!\O! &
\l    & \verb!\l! &
\L    & \verb!\L! &
\i    & \verb!\i! \\
\j    & \verb!\j! &
!`    & \verb!~`! &
?`    & \verb!?`! & \\
\hline
\end{tabular}
\end{center}

Além disso, os caracteres especiais mais comuns são:

\begin{center}
  \begin{tabular}{cccccccccc}
    caracter & \# & \$ & \% & \^{} & \& & \_ & \{ & \} & \~{} \\
    comando & \verb|\#| & \verb|\$| & \verb|\%| & \verb|\^| & \verb|\&| & \verb|\_| & \verb|\{| & \verb|\}| & \verb|\~|
  \end{tabular}
\end{center}

Para mostrar \textbackslash digite \verb|\textbackslash|.

Outros dois símbolos muito usados são:

\begin{center}
  \begin{tabular}{|l|l|}
    1\textordmasculine & \verb|\textordmasculine| \\
    1\textordfeminine & \verb|\textordfeminine| \\
  \end{tabular}
\end{center}

\noindent experimente também \verb|$60^\circ$|.

O LaTeX possui milhares de símbolos e eles estão documentados em \href{http://linorg.usp.br/CTAN/info/symbols/comprehensive/symbols-a4.pdf}{The Comprehensive \LaTeX\ Symbol List} de Scott Pakin. Você também pode digitar \verb|texdoc symbols| no terminal para visualizar a documentação. Vejamos alguns exemplos descritos nesse manual:

\begin{center}
  \begin{tabular}{|l|l|}
    @ & \verb|@|\\
    \copyright & \verb|\copyright| \\
    \textregistered & \verb|\textregistered|
  \end{tabular}
\end{center}

Às vezes é necessário carregar um pacote específico para mais símbolos.

O símbolo \textperthousand{} requer o pacote \texttt{textcomp} (\verb|texdoc textcomp|).

\begin{verbatim}
\usepackage{textcomp}
\begin{document}
 \textperthousand
\end{document}
\end{verbatim}

Veja alguns símbolos do pacote \texttt{marvosym} (\verb|texdoc marvosym|).

\begin{center}
  \begin{tabular}{|l|l|}
    \MVAt & \verb|\MVAt| \\
    \Smiley & \verb|\Smiley| \\
    \Heart & \verb|\Heart| \\
    \Radioactivity & \verb|\Radioactivity| \\
    \Football & \verb|\Football| \\
    \NoIroning & \verb|\NoIroning| \\
    \MartinVogel & \verb|\MartinVogel|
  \end{tabular}
\end{center}

Pacote \texttt{wasysym} (\verb|texdoc wasysym|).

\begin{center}
  \begin{tabular}{|l|l|}
    \smiley & \verb|\smiley| \\
    \male & \verb|\male| \\
    \female & \verb|\female| \\
    \eighthnote & \verb|\eighthnote| \\
    \twonotes & \verb|\twonotes| \\
  \end{tabular}
\end{center}

O pacote \texttt{eurosym} oferece os símbolos do Euro.

\begin{center}
  \begin{tabular}{|l|l|}
    \officialeuro & \verb|\officialeuro| \\
    \geneuro & \verb|\geneuro| \\
    \geneuronarrow & \verb|\geneuronarrow| \\
    \geneurowide & \verb|\geneurowide| \\
  \end{tabular}
\end{center}

Atenção: se você carregar todos os pacotes mencionados de uma vez eles devem respeitar uma ordem para que não gere conflito; coloque o \texttt{marvosym} primeiro.

\begin{verbatim}
\usepackage{textcomp}
\usepackage{marvosym}
\usepackage{wasysym}
\usepackage{eurosym}
\end{verbatim} 

Veja em \href{}{simbolos.tex} alguns exemplos de símbolos.

\subsection*{Símbolos matemáticos}

A seguir alguns símbolos matemáticos. Se for colocar numa linha de texto lembre-se de usar \verb|$...$|. Requer o pacote \texttt{amssymb}.

\begin{center}
  \begin{tabular}{|l|l|l|l|l|l}
    $\leqslant$ & \verb|\leqslant| 	& $\bot$ & \verb|\bot| 		& $\infty$ & \verb|\infty| \\
    $\geqslant$ & \verb|\geqslant| 	& $\exists$ & \verb|\exists| 	& $\therefore$ & \verb|\therefore| \\
    $\rightarrow$ & \verb|\rightarrow| 	& $\forall$ & \verb|\forall| 	& $\triangle$ & \verb|\triangle| \\
    $\Rightarrow$ & \verb|\Rightarrow| 	& $\in$ & \verb|\in| 		& $\overline{a}$ & \verb|\overline{a}| \\
    $\Leftrightarrow$ & \verb|\Leftrightarrow| & $\emptyset$ & \verb|\emptyset| & $\overrightarrow{a}$ & \verb|\overrightarrow{a}| \\
  \end{tabular}
\end{center}

No site \href{http://detexify.kirelabs.org/classify.html}{Detexify} você pode desenhar o símbolo que ele descobre o código LaTeX pra você.

\section{Enfatizando um texto}

O LaTeX enfatiza um texto de uma forma bem simples: se o texto estiver no modo normal ele enfatiza em \emph{itálico}, caso contrário (texto em itálico) ele enfatiza para o modo normal.

O comando é \verb|\emph{}|. Veja um exemplo:

\verb|Veja este texto \emph{enfatizado}.|

% exemplo no site: centralizado e com estilo.
\begin{center}
Veja este texto \emph{enfatizado}.                                  \end{center}

\verb|\emph{Se o texto já estiver enfatizado a fonte \emph{normal} será usada para|

\verb|enfatizar este texto.}|

\begin{center}
\emph{Se o texto já estiver enfatizado a fonte \emph{normal} será usada para enfatizar este texto.}
\end{center}

\section{Família (family)}

%fonte_familia.tex

Essencialmente é o \emph{tipo de fonte}, ou um grupo de fontes que possuem o mesmo \emph{estilo}.

Conforme dito antes, a fonte padrão do LaTeX é a \href{http://en.wikipedia.org/wiki/Computer_Modern}{Computer Modern}. E este possui $3$ estilos diferentes, conforme a tabela abaixo:

% <div align=center>
% <table align=center border=1 cellpadding=2 style=border-collapse: collapse;><tbody>
% <tr> <td><strong>comando</strong></td> 
% <td><strong>equivalente</strong></td> 
% <td><strong>valor*</strong></td> 
% <td><strong>resultado</strong></td>
% </tr>
% <tr>
% <td><span style=font-family: 'Courier New'; font-size: x-small;>\textrm{…}</span></td> 
% <td><span style=font-family: 'Courier New';>{\rmfamily…}</span></td> 
% <td><span style=font-family: 'Courier New';>cmr</span></td> <td>Romano (padrão)</td>
% </tr>
% <tr>
% <td><span style=font-family: 'Courier New';>\textsf{…}</span></td> <td><span style=font-family: 'Courier New';>{\sffamily…}</span></td> <td><span style=font-family: 'Courier New';>cmss</span></td> <td>Sem serifa</td>
% </tr>
% <tr>
% <td><span style=font-family: 'Courier New';>\texttt{…}</span></td> <td><span style=font-family: 'Courier New';>{\ttfamily…}</span></td> <td><span style=font-family: 'Courier New';>cmtt</span></td> <td>Monoespaço (<span style=font-family: 'Courier New';>máquina de escrever)</span></td></tr>
% </tbody></table>
% </div>
% * Valor na classe \emph{article}.

\begin{center}
  \begin{tabular}{l|l|l|l}
    \textbf{comando} & \textbf{equivalente} & \textbf{valor*} & \textbf{resultado}\\
    \hline
    \verb|\textrm{...}| & \verb|{\rmfamily ...}| & \texttt{cmr} & romano (padrão)\\
    \verb|\textsf{...}| & \verb|{\sffamily ...}| & \texttt{cmss} & \textsf{sem serifa}\\
    \verb|\texttt{...}| & \verb|{\ttfamily ...}| & \texttt{cmtt} & \texttt{monoespaço (máquina de escrever)} \\
  \end{tabular}
\end{center}

\noindent * valor na classe \emph{article}.

\textbf{Obs}: os comandos da primeira coluna são mais recomendados que os da segunda coluna, pois o resultado nem sempre é o mesmo em relação ao espaçamento.

Veja no exemplo a seguir como usar a fonte sem serifa num \textbf{trecho de texto}.

\

% site: código + texto sem serifa

\begin{minipage}[t]{.45\textwidth}
  \begin{verbatim}
  \textsf{Texto sem serifa}
  \end{verbatim}
\end{minipage}
\begin{minipage}[t]{.45\textwidth}
  \textsf{Texto sem serifa}
\end{minipage}

Também pode ser escrito da seguinte forma:

\begin{minipage}[t]{.45\textwidth}
  \begin{verbatim}
  {\sffamily Texto sem serifa}
  \end{verbatim}
\end{minipage}
\begin{minipage}[t]{.45\textwidth}
  {\sffamily Texto sem serifa}
\end{minipage}

Para usar uma família no \textbf{documento todo} digite no preâmbulo

\begin{verbatim}
\renewcommand\familydefault{\sfdefault}
\end{verbatim} 

\noindent e continue o texto normalmente. Baixe o \href{}{fonte\_sem\_serifa.tex}.

Muitas fontes no LaTeX requer algum pacote no preâmbulo, mas outras já são carregadas por padrão. Veja a seguir as famílias de fontes mais comuns segundo o \href{http://linorg.usp.br/CTAN/macros/latex/doc/fntguide.pdf}{\LaTeXe font selection} (\verb|texdoc fntguide|).

\begin{center}
  \begin{tabular}{ll}
    \textbf{valor} & \textbf{família} \\
    \texttt{cmr} & Computer Modern Roman \\
    \texttt{cmss} & Computer Modern Sans \\
    \texttt{cmtt} & Computer Modern Typewriter \\
    \texttt{cmm} & Computer Modern Math Italic \\
    \texttt{cmsy} & Computer Modern Math Symbols \\
    \texttt{cmex} & Computer Modern Math Extensions \\
    \texttt{ptm} & Adobe Times \\
    \texttt{phv} & Adobe Helvetica \\
    \texttt{pcr} & Adobe Courier \\
  \end{tabular}
\end{center}

Vejamos um exemplo com a família \emph{Adobe Helvetica} num \textbf{trecho de texto}.

\begin{minipage}[t]{.75\textwidth}
  \begin{verbatim}
  {\fontfamily{phv}\selectfont Texto Helvetica}
  \end{verbatim}
\end{minipage}
\begin{minipage}[t]{.20\textwidth}
  {\fontfamily{phv}\selectfont Texto Helvetica}
\end{minipage}

Repare no uso de \verb|\selectfont|, sem ele você pode obter resultados inesperados. Veja também as chaves \verb|{...}| nos extremos, que seleciona o texto desejado, sem ele todo o texto após o comando teria a mesma aparência.

Segundo o \href{http://linorg.usp.br/CTAN/macros/latex/doc/fntguide.pdf}{fntguide} podemos usar o comando \verb|\usefont| para suprimir o uso do \verb|\selectfont|. Veremos sua sintaxe mais adiante. No exemplo anterior, o comando fica assim:

\begin{verbatim}
{\usefont{T1}{phv}{}{} Texto Helvetica}
\end{verbatim}

Uma sugestão também é usar um novo comando.

\begin{verbatim}
%familia
\newcommand{\fonte}[2]{
  {\fontfamily{#1}\selectfont #2}
}
\begin{document}
  \fonte{phv}{Texto Helvetica}

  \fonte{pcr}{Texto Adobe Courier}
\end{document}
\end{verbatim}

E ainda, com este novo comando podemos usar qualquer outra família, tornando mais flexível a manipulação das fontes.

\textbf{Sugestão}: Use \verb|\newcommand| sempre que possível, assim você facilita e muito na estruturação lógica do documento.	

Ainda temos mais uma opção, segundo Alan Munn em \href{http://tex.stackexchange.com/a/25251/1940}{Restricting the scope of the selection (TeX.sx)}: definir um novo ambiente.

\begin{verbatim}
%definindo novo ambiente
\newenvironment{fonteHelvetica}{\fontfamily{phv}\selectfont}{\par}
%usando o ambiente
\begin{fonteHelvetica}
  Texto Helvetica num ambiente
\end{fonteHelvetica}
\end{verbatim}

Para usar a fonte Helvetica no \textbf{documento todo} temos duas opções:

\begin{enumerate}[1)]
 \item Renomeando a fonte padrão

\begin{verbatim}
%Adobe Helvetica
\renewcommand\familydefault{phv}
\end{verbatim}

ou

 \item carregando o pacote \href{http://ctan.tche.br/help/Catalogue/entries/helvet.html}{helvet}.

\begin{verbatim}
\usepackage{helvet}
\renewcommand\familydefault{\sfdefault} 
\end{verbatim} 
\end{enumerate}

A vantagem é que, em ambos os casos, podemos digitar o texto no modo normal, \emph{itálico} e/ou \textbf{negrito} que teremos todos os formatos do Helvetica (sem serifa). Baixe \href{}{fonte\_familia\_helvetica\_all1.tex} e \href{}{fonte\_familia\_helvetica\_all2.tex}

Veja a seguir algumas das $35$ fontes \emph{postscript} suportadas pela distribuição básica \href{http://ctan.tche.br/macros/latex/required/psnfss/psnfss2e.pdf}{psnfss2e} (texdoc) e relacionadas em \href{http://www.tug.org/fontname/html/Standard-PostScript-fonts.html}{PostScript-fonts} (texdoc).

%/usr/local/texlive/2011/texmf-dist/doc/fonts/fontname

\begin{verbatim}
% New Century Schoolbook
\fonte{pnc}{\frase}

% Uncial
\fonte{uncl}{\frase}

% Zapf Chancery
\fonte{pzc}{\frase}
\end{verbatim} 

% New Century Schoolbook
\fonte{pnc}{\frase}

% Uncial
\fonte{uncl}{\frase}

% Zapf Chancery
\fonte{pzc}{\frase}

% site: figfamilia_psnfss2e.png

\

Baixe \href{}{fonte\_familia\_psnfss2e.tex} e compare \emph{Computer Modern Roman} com \emph{Times}.

Observe ainda, que neste exemplo não foi necessário carregar nenhum pacote de fonte.

\section{Séries (series)}

Basicamente é a fonte em \textbf{negrito}, mas, segundo o \href{http://linorg.usp.br/CTAN/macros/latex/doc/fntguide.pdf}{fntguide}, existem 5 séries diferentes, são eles:

\begin{center}
  \begin{tabular}{l|l|l|l}
    \textbf{comando} & \textbf{equivalente} & \textbf{valor} & \textbf{resultado}\\
    \hline
    \verb|\textbf{...}| & \verb|{\bfseries ...}| & \texttt{bx} & \textbf{negrito (bold extended)}\\
    \verb|\textmd{...}| & \verb|{\mdseries ...}| & \texttt{m} & \textmd{negrito médio}\\
			& 			 & \texttt{b} & negrito (bold) \\
			& 			 & \texttt{sb} & semi-negrito \\
			& 			 & \texttt{c} & condensado \\
  \end{tabular}
\end{center}

Veja no exemplo a seguir um \textbf{trecho de texto} em negrito:

\

\begin{minipage}[t]{.5\textwidth}
  \begin{verbatim}
  \textbf{Texto em negrito}
  \end{verbatim}
\end{minipage}
\begin{minipage}[t]{.5\textwidth}
  \textbf{Texto em negrito}
\end{minipage}

Por padrão a classe \emph{article} aceita os comandos \verb|\textbf{...}| (bx) e \verb|\textmd{...}| (m). O efeito do segundo é idêntico a fonte no modo normal.

As demais \emph{séries} geralmente são aceitas em fontes \emph{PostScript}.

\subsection*{Sublinhado}
%wikibooks #fonts

A fonte \uline{sublinhada} não faz parte da classe \emph{article}, para usá-la deve-se carregar o pacote \texttt{ulem} junto com a opção \verb|normalem|, que mantém o texto no modo normal por padrão.

\begin{verbatim}
\usepackage[normalem]{ulem}
\end{verbatim}

Veja a seguir um texto sublinhado.

\begin{minipage}[t]{.5\textwidth}
  \begin{verbatim}
  \uline{Texto sublinhado}
  \end{verbatim}
\end{minipage}
\begin{minipage}[t]{.5\textwidth}
  \uline{Texto sublinhado}
\end{minipage}

Veja a seguir os 7 modos de texto sublinhado (\verb|texdoc ulem|):

\begin{center}
  \begin{tabular}{ll}
    \textbf{comando} & \textbf{resultado} \\
    \verb|\uline{...}| 	& \uline{sublinhado} \\
    \verb|\uuline{...}| 	& \uuline{sublinhado duplo} \\
    \verb|\uwave{...}| 	& \uwave{ondulado} \\
    \verb|\sout{...}|	 	& \sout{riscado} \\
    \verb|\xout{...}|	 	& \xout{hachurado} \\
    \verb|\dashuline{...}| 	& \dashuline{tracejado} \\
    \verb|\dotuline{...}| 	& \dotuline{pontilhado} \\
  \end{tabular}
\end{center}

% site: figulem.png

Baixe \href{}{fonte\_sublinhada.tex}

\section{Formas (shape)}

As 5 \emph{formas} de fontes são:

\begin{center}
  \begin{tabular}{l|l|l|l}
    \textbf{comando} & \textbf{equivalente} & \textbf{valor} & \textbf{resultado}\\
    \hline
    \verb|\textup{...}| & \verb|{\upshape ...}| & \texttt{n} 	& \textup{forma vertical}\\
    \verb|\textit{...}| & \verb|{\itshape ...}| & \verb|\it| 	& \textit{itálico}\\
    \verb|\textsl{...}| & \verb|{\slshape ...}| & \verb|\sl| 	& \textsl{inclinado}\\
    \verb|\textsc{...}| & \verb|{\scshape ...}| & \verb|\sc| 	& \textsc{caixa baixa}\\
    \verb|\uppercase{...}| & 			& 		& \uppercase{caixa alta}\\
  \end{tabular}
\end{center}

Veja o significado de cada sigla:

\begin{center}
  \begin{tabular}{ll}
    \textbf{valor} & \textbf{forma} \\
    \verb|n| 	& normal \\
    \verb|it| 	& italic (itálico) \\
    \verb|sl| 	& slanted (inclinado) \\
    \verb|sc| 	& small caps (capital pequeno) \\
  \end{tabular}
\end{center}

Veja os exemplos a seguir.

\

\begin{minipage}[t]{.5\textwidth}
  \begin{verbatim}
  \textit{Texto itálico}
  ou
  {\it Texto itálico}
  \end{verbatim}
\end{minipage}
\begin{minipage}[t]{.5\textwidth}
  \textit{Texto itálico}
\end{minipage}

\begin{minipage}[t]{.5\textwidth}
  \begin{verbatim}
  \textsl{Texto inclinado}
  ou
  {\sl Texto inclinado}
  \end{verbatim}
\end{minipage}
\begin{minipage}[t]{.5\textwidth}
  \textsl{Texto inclinado}
\end{minipage}

\begin{minipage}[t]{.5\textwidth}
  \begin{verbatim}
  \textsc{caixa baixa}
  ou
  {\sc caixa baixa}
  \end{verbatim}
\end{minipage}
\begin{minipage}[t]{.5\textwidth}
  \textsc{caixa baixa}
\end{minipage}

\begin{minipage}[t]{.5\textwidth}
  \begin{verbatim}
  \uppercase{caixa alta}
  \end{verbatim}
\end{minipage}
\begin{minipage}[t]{.5\textwidth}
  \uppercase{caixa alta}
\end{minipage}

Baixe \href{}{fonte\_shape.tex}

Para usar \textit{itálico} no \textbf{documento todo} use

\begin{verbatim}
\renewcommand{\shapedefault}{\itdefault}
\end{verbatim}

\section{Tamanho (size)}

A quarta característica essencial de uma fonte é o \emph{tamanho}.

Tanto em \emph{article, report} ou \emph{book} o tamanho de fonte padrão é 10pt, $(1pt \cong 0.3513\,mm)$ considerada como tamanho \emph{normal}. Além disso, cada classe aceita como opção 10pt, 11pt ou 12pt como tamanho normal.

A partir daí cada tamanho de fonte é definido por um nome. Veja na tabela a seguir cada nome e seu valor numérico equivalente.

\begin{center}
  \begin{tabular}{ll}
    \textbf{comando} & \textbf{valor*} \\
    \verb|\tiny| 		& 5 pt \\
    \verb|\scriptsize| 		& 7 pt \\
    \verb|\footnotesize| 	& 8 pt \\
    \verb|\small| 		& 9 pt \\
    \verb|\normalsize| 		& 10 pt (padrão) \\
    \verb|\large| 		& 12 pt \\
    \verb|\Large| 		& 14.4 pt \\
    \verb|\LARGE| 		& 17.28 pt \\
    \verb|\huge| 		& 20.74 pt \\
    \verb|\Huge| 		& 24.88 pt \\
  \end{tabular}
\end{center}

* Estes são os valores para a classe \emph{article} com tamanho padrão 10pt. Para 11pt e 12pt os valores variam um pouco. Mais informações em \href{http://en.wikibooks.org/wiki/LaTeX/Text_Formatting#Sizing_text}{wikibooks sizing text}.

Veja no exemplo a seguir como usar um dos tamanhos de fonte.

\begin{minipage}[t]{.5\textwidth}
  \begin{verbatim}
  {\Huge Texto tamanho Huge}
  \end{verbatim}
\end{minipage}
\begin{minipage}[t]{.5\textwidth}
  {\Huge Texto tamanho Huge}
\end{minipage}

Também poderíamos definir um novo comando, por exemplo

\verb|\newcommand{\fontegrande}[1]{{\Huge #1}}| e usá-lo

\verb|\fontegrande{Texto grande}|

Baixe \href{}{fonte\_tamanho.tex}.

Observe o uso de chave dupla \verb|{{...}}| isto é necessário para que o tamanho da fonte seja aplicado somente no texto selecionado.

\subsection*{Definindo o tamanho da fonte numericamente}

Uma outra forma de definir o tamanho da fonte é numericamente com o comando

\begin{verbatim}
\fontsize{size}{baselineskip}
\end{verbatim}

\noindent mas este comando não é muito usual.

\emph{size} significa tamanho e \emph{baselineskip} é a distância vertical entre linhas num parágrafo. Por padrão \verb|baselineskip = 1.2 * size|, assim para uma fonte de 10pt teremos

\begin{verbatim}
{\fontsize{10}{12}\selectfont Texto tamanho 10pt}
\end{verbatim}

É aconselhável que se use um novo comando colocando o \emph{baselineskip} como opção.

\begin{verbatim}
\newcommand{\tamanho}[3][12pt]{
  {\fontsize{#2}{#1}\selectfont #3}
}
\end{verbatim}

\noindent onde \verb|#1| é o \emph{baselineskip} definido por padrão como 12pt (\verb|#1| é opcional, neste caso). \verb|#2| é size (tamanho) e \verb|#3| é o texto.

Veja no exemplo a seguir:

\begin{verbatim}
\tamanho[27.6]{23}{Tamanho de fonte definido por novo comando}
\end{verbatim}

Depois, digite outro tamanho de fonte para alternar, por exemplo:

\begin{verbatim}
\normalsize 
\end{verbatim}

Baixe \href{}{fonte\_tamanho.tex}.

\subsection*{O comando \texttt{\textbackslash usefont}}

A vantagem de uso do comando \verb|\usefont| é que ele suprime a necessidade do \verb|\selectfont|, mas ele não define o tamanho da fonte. Vejamos sua sintaxe:

\begin{verbatim}
\usefont{codificacao}{familia}{series}{forma}
\end{verbatim}

\noindent \emph{codificacao} - de saída, usualmente T1;

\noindent \emph{familia} - aceita a família abreviada (do campo valor*);

\noindent \emph{series} - \verb|m, b, bx, sb, c|;

\noindent \emph{forma} - \verb|n, it, sl, sc|.

Alguns campos podem ficar em branco, mas somente os últimos, ou seja, \verb|codificacao| é obrigatório, depois podemos preencher os próximos sem deixar campo em branco entre o primeiro e o último.

Exemplos:

\verb|{\usefont{T1}{}{}{} texto}| - mantém a forma padrão usada no momento.

\verb|{\usefont{T1}{cmr}{bx}{} texto}| - Computer Modern Roman negrito

\verb|{\usefont{T1}{phv}{m}{it} texto}| - Helvetica itálico

Neste caso, como o campo \emph{series} não pode ficar em branco, usamos a opção \verb|m| que não surte efeito nesta fonte.

Baixe \href{}{usefont.tex}.

\subsection*{O comando \texttt{\textbackslash DeclareFixedFont}}

Uma outra forma é declarando fontes estáticas, ou seja, as fontes sempre serão usadas na forma em que ela for declarada, sem possibilidade de variações.

\begin{verbatim}
\DeclareFixedFont{comando}{codificacao}{familia}{series}{forma}{tamanho}
\end{verbatim}

\noindent onde \verb|comando| é o nome do comando. Vejamos um exemplo:

\begin{verbatim}
\DeclareFixedFont{\fonteUtopiaItalico}{T1}{put}{m}{it}{10}
\end{verbatim}

{\fonteUtopiaItalico \frase}

\

Baixe \href{}{declarefixedfont.tex}

\section{Fontes coloridas}

ESCREVER SOBRE CORES NAS FONTES.

\section{Outras fontes}

Veja alguns sites com mais fontes para o \LaTeX:

%listar os sites de fontes

Mais adiante veremos como usar as fontes do \href{http://www.tug.dk/FontCatalogue/}{catálogo de fontes do \LaTeX}. Neste tópico veremos como usar as fontes descritas em \href{http://suppiya.files.wordpress.com/2008/02/latex_fonts.pdf}{LaTeX Fonts} de Ki-Joo Kim.

Basicamente precisamos saber somente qual é a família de cada fonte.

Caso a fonte não esteja instalada \href{http://latexbr.blogspot.com.br/2011/10/atualizando-os-pacotes-do-texlive-2011.html}{atualize} o seu TeXLive.

Para aplicar as fontes usaremos o comando \verb|\fonte{}{}| mencionado no início deste material, junto com nossa \verb|\frase| de exemplo. E outros casos, o comando segue em cada exemplo.

% latex_fonts

\subsection*{Computer Modern Roman (padrão)}

\frase

\subsection*{antt}

\verb|\usepackage{anttor}|

\verb|\fonte{antt}{\frase}|

\fonte{antt}{\frase}

\verb|\emph{\fonte{antt}{\frase}}|

\emph{\fonte{antt}{\frase}}


\subsection*{augie}

Não precisa carregar pacote.

\verb|\fonte{augie}{\frase}|

\fonte{augie}{\frase}


\subsection*{brushscr}

\verb|\usepackage[T1]{pbsi}|

\verb|\textbsi{\frase}|

\textbsi{\frase}


\subsection*{calligra}

\verb|\usepackage{calligra}|

\verb|\fonte{calligra}{\frase}|

\fonte{calligra}{\frase}


\subsection*{ifsym}

\verb|\usepackage[clock,weather]{ifsym}|

\textbf{Obs}: O pacote \texttt{ifsym} não será mostrado aqui porque dá conflito com \texttt{marvosym}.

\texttt{ifsym} é um pacote de símbolos, dentre eles: \verb|\emph{clock}| que mostra as horas: \verb|\showclock{3}{40}|.

\verb|\SunCloud| que mostra um sol com nuvem e \verb|\textifsym{123.45}|.

Para ver os demais símbolos digite \verb|texdoc ifsym| no terminal.


\subsection*{concrete}

\verb|\usepackage{concrete}|

\verb|\fonte{ccr}{\frase}|

\verb|\emph{\fonte{ccr}{\frase}}|


\subsection*{iwona}

\verb|\usepackage[math]{iwona}|

Este pacote possui suporte para modo matemático.

\verb|\fonte{iwona}{\frase}|

\textbf{Obs}: Veja o resultado de \textit{concrete} e \textit{iwona} no arquivo de exemplo.

\subsection*{yfonts}

\verb|\usepackage{yfonts}|

\textbf{Gotisch} - \verb|\textgoth{\frase}|

\textgoth{\frase}

\textbf{Schwabacher} - \verb|\textswab{\frase}|

\textswab{\frase}

\textbf{Fraktur} - \verb|\textfrak{\frase}|

\textfrak{\frase}

\textbf{Baroque} - \verb|\textinit{A B C D E}|

\textinit{A B C D E}

\yinipar{L}etra grande ao iniciar uma frase. Uma frase grande com pelo menos 4 linhas de texto. Este tipo de fonte era usado em livros antigos, e até hoje ainda é usado em livros sagrados ou clássicos. Esta fonte dá um estilo decorativo a frase. Para escrevê-lo digite, por exemplo, \verb|\yinipar{L}etra|. Se o parágrafo tiver várias linhas de texto a letra quebra as linhas iniciais do texto.

Baixe \href{}{latex\_fonts.tex}


\section{Fontes matemáticas}

Veremos apenas o básico das fontes matemáticas. Lembrando que é necessário usar \verb|$...$|.

As fontes matemáticas mais comuns são:

\begin{center}
  \begin{tabular}{l|l}
    \textbf{comando} & \textbf{resultado}\\
    \hline
    \verb|\mathcal{...}| & $\mathcal{ABXYZ}$\\
    \verb|\mathbb{...}|* & $\mathbb{ZRC}$\\
    \verb|\mathbf{...}| & $\mathbf{uv}$\\
    \verb|\mathfrak{...}| & $\mathfrak{ABXYZ}$\\
    \verb|\mathtt{...}| & $\mathtt{ABXYZ}$\\
    \verb|\mathscr{...}|* & $\mathscr{ABXYZ}$\\
  \end{tabular}
\end{center}

Alfabeto com \verb|\mathcal{...}|. Somente letra maiúscula.

\

\begin{center}
$\mathcal{ABCDEFGHIJKLMNOPQRSTUVWXYZ}$                    
\end{center}

\

Conjuntos numéricos com \verb|\mathbb{...}|. Requer o pacote \texttt{amsfonts}.

\

\begin{center}
$\mathbb{NZQRC}$
\end{center}

\

Exemplo:

\begin{minipage}[t]{.6\textwidth}
  \begin{verbatim}
$\mathcal{U}: \mathbb{R} \to \mathbb{C}$
  \end{verbatim}
\end{minipage}
\begin{minipage}[t]{.5\textwidth}
  $\mathcal{U}: \mathbb{R} \to \mathbb{C}$
\end{minipage}

O comando \verb|\mathbf{...}| escreve em negrito.

\

\begin{center}
$\mathbf{ABCDEFGHIJKLMNOPQRSTUVWXYZ}$

$\mathbf{abcdefghijklmnopqrstuvwxyz}$
\end{center}
\


Exemplo:

\begin{verbatim}
   $\vec{\mathbf{v}} = x\mathbf{i} + y\mathbf{j} + z\mathbf{k}$
\end{verbatim}

$\vec{\mathbf{v}} = x\mathbf{i} + y\mathbf{j} + z\mathbf{k}$

\

O comando \verb|\mathfrak{...}|.

\

\begin{center}
$\mathfrak{ABCDEFGHIJKLMNOPQRSTUVWXYZ}$

$\mathfrak{abcdefghijklmnopqrstuvwxyz}$
\end{center}
\

Exemplo:

\begin{minipage}[t]{.5\textwidth}
  \begin{verbatim}
$\mathfrak{U} \subset \mathfrak{C}$
  \end{verbatim}
\end{minipage}
\begin{minipage}[t]{.5\textwidth}
  $\mathfrak{U} \subset \mathfrak{C}$
\end{minipage}


\

O comando \verb|\mathtt{...}|.

\

\begin{center}
$\mathtt{ABCDEFGHIJKLMNOPQRSTUVWXYZ}$

$\mathtt{abcdefghijklmnopqrstuvwxyz}$
\end{center}

\

O comando \verb|\mathscr{...}| requer o pacote \texttt{mathrsfs}. Somente letra maiúscula.

\

\begin{center}
$\mathscr{ABCDEFGHIJKLMNOPQRSTUVWXYZ}$
\end{center}

\

Exemplo:

\begin{verbatim}
$\int \mathscr{L} dt = \mathscr{F} + \mathscr{H} + \mathscr{U}$
\end{verbatim}

$\int \mathscr{L} dt = \mathscr{F} + \mathscr{H} + \mathscr{U}$

\

Veja também \href{}{fonte\_mathpazo.tex}

\subsection*{Fontes matemáticas em apresentações com Beamer}

O \textit{beamer}, por padrão, usa fontes sem serifa, então se você quiser que suas expressões matemáticas tenham serifa, digite no preâmbulo

\begin{verbatim}
\usefonttheme[onlymath]{serif}
\end{verbatim}

\subsection*{Tamanho de fonte em expressões matemáticas}

O tamanho da fonte em expressões matemáticas é sempre padrão, mas caso a expressão matemática fique muito grande, excedendo a largura da página, é possível reduzir seu tamanho com o comando \verb|\scriptstyle| ou \verb|\scriptscriptstyle|. Este comando deve ser digitado no início da expressão.

$p(x) = a_0 + a_1 x + a_2 x^2 + \ldots + a_n x^n$

$\scriptstyle p(x) = a_0 + a_1 x + a_2 x^2 + \ldots + a_n x^n$

$\scriptscriptstyle p(x) = a_0 + a_1 x + a_2 x^2 + \ldots + a_n x^n$

Veja outros tamanhos:

\newcommand{\funcaoint}{
  f(t) = \frac{T}{2\pi} \int{\frac{1}{\sin\frac{\omega}{t}}}dt
}

\begin{center}
  \begin{tabular}{l|l}
    \textbf{modo} & \textbf{resultado}\\
    \hline
    padrão & $\funcaoint$\\
    \verb|\displaystyle| & $\displaystyle\funcaoint$\\
    \verb|\scriptstyle| & $\scriptstyle\funcaoint$\\
    \verb|\scriptscriptstyle| & $\scriptscriptstyle\funcaoint$\\
    \verb|\textstyle| & $\textstyle\funcaoint$\\
  \end{tabular}
\end{center}

Baixe \href{}{fonte\_mat\_tamanho.tex}

\section{Fontes grandes}

Como vimos anteriormente, o tamanho de fonte vai até 24.88 pt (\verb|\Huge|). Então para fontes maiores que isso precisamos do pacote \texttt{fix-cm}. Veja o exemplo a seguir.

\begin{verbatim}
{\fontsize{80}{100}\selectfont Texto grande}
\end{verbatim}

{\fontsize{80}{100}\selectfont Texto grande}

Para expressões matemáticas usamos o pacote \texttt{exscale}. Este pacote corrige o tamanho de todos os elementos da expressão matemática.

\begin{minipage}[t]{0.5\textwidth}
\begin{verbatim}
{\Huge
\[
  \int_0^1 {\frac{\sqrt x}{2}dx} 
\]
}
\end{verbatim} 
\end{minipage} 
\begin{minipage}[t]{0.5\textwidth}
{\Huge
\[
  \int_0^1 {\frac{\sqrt x}{2}dx} 
\]
} 
\end{minipage} 

Para fontes maiores ainda usamos o pacote \texttt{graphicx} que permite uma escala na expressão matemática em qualquer tamanho. Use o comando \verb|\scalebox{ }|.

\begin{verbatim}
\[
  \scalebox{5}{
    $\displaystyle\int_0^1 {\frac{\sqrt x}{2}dx}$
  }
\]
\end{verbatim}
\[
  \scalebox{5}{
    $\displaystyle\int_0^1 {\frac{\sqrt x}{2}dx}$
  }
\]

Baixe {fonte\_grande.tex}.



\section{Alfabeto grego no modo texto}

É comum usar o alfabeto grego no modo matemático, mas com o pacote \href{http://ctan.tche.br/macros/latex/contrib/textgreek/textgreek.pdf}{\texttt{textgreek}} é possível inserir as letras gregas no modo texto. Veja um exemplo:

\verb|\textalpha, \textbeta, \textgamma|



\section{pifont}

\texttt{pifont} é um pacote que usa símbolos com uma funcionalidade especial.

Digitando \verb|\ding{38}| retorna \ding{38}. Veja mais dois exemplos: \ding{73} 1978 \ding{61} 2012.

Com o ambiente \texttt{dinglist} podemos criar uma lista personalizada.

\

\begin{minipage}[T]{.45\textwidth}
\begin{verbatim}
\begin{dinglist}{43}
 \item um
 \item dois
 \item tres
\end{dinglist}
\end{verbatim}
\end{minipage}
\begin{minipage}[T]{.45\textwidth}
\begin{dinglist}{43}
 \item um
 \item dois
 \item tres
\end{dinglist}
\end{minipage}

\

\

\verb|\begin{dinglist}{51}|

\begin{dinglist}{51}
 \item 3 ovos
\end{dinglist}

\verb|\begin{dinglist}{55}|

\begin{dinglist}{55}
 \item 100 g de sal
\end{dinglist}

Com o comando \texttt{dingautolist} podemos criar uma lista onde a numeração dos símbolos aumenta a cada item.

\begin{minipage}[T]{.45\textwidth}
\begin{verbatim}
\begin{dingautolist}{192}
 \item um
 \item dois
 \item três
 \item quatro
 \item cinco
 \item seis
 \item sete
 \item oito
 \item nove
 \item dez
\end{dingautolist}
\end{verbatim}
\end{minipage}
\begin{minipage}[T]{.45\textwidth}
\begin{dingautolist}{192}
 \item um
 \item dois
 \item três
 \item quatro
 \item cinco
 \item seis
 \item sete
 \item oito
 \item nove
 \item dez
\end{dingautolist}
\end{minipage}

\newpage 

Veja a tabela com os símbolos do \texttt{pifont}.

\begin{table}[h]
 \centering
  \begin{tabular}{|cc|cc|cc|cc|cc|cc|cc|cc|}
  \hline
32	&	\ding{32}	& 33	&	\ding{33}	&	34	&	\ding{34}	&	35	&	\ding{35}	&	36	&	\ding{36}	&	37	&	\ding{37}	&	38	&	\ding{38}	&	39	& \ding{39} \\ \hline
40	&	\ding{40}	&	41	&	\ding{41}	&	42	&	\ding{42}	&	43	&	\ding{43}	&	44	&	\ding{44}	&	45	&	\ding{45}	&	46	&	\ding{46}	&	47	&	\ding{47}  \\ \hline
48	&	\ding{48}	&	49	&	\ding{49}	&	50	&	\ding{50}	&	51	&	\ding{51}	&	52	&	\ding{52}	&	53	&	\ding{53}	&	54	&	\ding{54}	&	55	&	\ding{55} \\ \hline
56	&	\ding{56}	&	57	&	\ding{57}	&	58	&	\ding{58}	&	59	&	\ding{59}	&	60	&	\ding{60}	&	61	&	\ding{61}	&	62	&	\ding{62}	&	63	&	\ding{63} \\ \hline
64	&	\ding{64}	&	65	&	\ding{65}	&	66	&	\ding{66}	&	67	&	\ding{67}	&	68	&	\ding{68}	&	69	&	\ding{69}	&	70	&	\ding{70}	&	71	&	\ding{71} \\ \hline
72	&	\ding{72}	&	73	&	\ding{73}	&	74	&	\ding{74}	&	75	&	\ding{75}	&	76	&	\ding{76}	&	77	&	\ding{77}	&	78	&	\ding{78}	&	79	&	\ding{79} \\ \hline
80	&	\ding{80}	&	81	&	\ding{81}	&	82	&	\ding{82}	&	83	&	\ding{83}	&	84	&	\ding{84}	&	85	&	\ding{85}	&	86	&	\ding{86}	&	87	&	\ding{87} \\ \hline
88	&	\ding{88}	&	89	&	\ding{89}	&	90	&	\ding{90}	&	91	&	\ding{91}	&	92	&	\ding{92}	&	93	&	\ding{93}	&	94	&	\ding{94}	&	95	&	\ding{95} \\ \hline
96	&	\ding{96}	&	97	&	\ding{97}	&	98	&	\ding{98}	&	99	&	\ding{99}	&	100	&	\ding{100}	&	101	&	\ding{101}	&	102	&	\ding{102}	&	103	&	\ding{103} \\ \hline
104	&	\ding{104}	&	105	&	\ding{105}	&	106	&	\ding{106}	&	107	&	\ding{107}	&	108	&	\ding{108}	&	109	&	\ding{109}	&	110	&	\ding{110}	&	111	&	\ding{111} \\ \hline
112	&	\ding{112}	&	113	&	\ding{113}	&	114	&	\ding{114}	&	115	&	\ding{115}	&	116	&	\ding{116}	&	117	&	\ding{117}	&	118	&	\ding{118}	&	119	&	\ding{119} \\ \hline
120	&	\ding{120}	&	121	&	\ding{121}	&	122	&	\ding{122}	&	123	&	\ding{123}	&	124	&	\ding{124}	&	125	&	\ding{125}	&	126	&	\ding{126}	&		&	 \\ \hline
	&		&	161	&	\ding{161}	&	162	&	\ding{162}	&	163	&	\ding{163}	&	164	&	\ding{164}	&	165	&	\ding{165}	&	166	&	\ding{166}	&	167	&	\ding{167} \\ \hline
168	&	\ding{168}	&	169	&	\ding{169}	&	170	&	\ding{170}	&	171	&	\ding{171}	&	172	&	\ding{172}	&	173	&	\ding{173}	&	174	&	\ding{174}	&	175	&	\ding{175} \\ \hline
176	&	\ding{176}	&	177	&	\ding{177}	&	178	&	\ding{178}	&	179	&	\ding{179}	&	180	&	\ding{180}	&	181	&	\ding{181}	&	182	&	\ding{182}	&	183	&	\ding{183} \\ \hline
184	&	\ding{184}	&	185	&	\ding{185}	&	186	&	\ding{186}	&	187	&	\ding{187}	&	188	&	\ding{188}	&	189	&	\ding{189}	&	190	&	\ding{190}	&	191	&	\ding{191} \\ \hline
192	&	\ding{192}	&	193	&	\ding{193}	&	194	&	\ding{194}	&	195	&	\ding{195}	&	196	&	\ding{196}	&	197	&	\ding{197}	&	198	&	\ding{198}	&	199	&	\ding{199} \\ \hline
200	&	\ding{200}	&	201	&	\ding{201}	&	202	&	\ding{202}	&	203	&	\ding{203}	&	204	&	\ding{204}	&	205	&	\ding{205}	&	206	&	\ding{206}	&	207	&	\ding{207} \\ \hline
208	&	\ding{208}	&	209	&	\ding{209}	&	210	&	\ding{210}	&	211	&	\ding{211}	&	212	&	\ding{212}	&	213	&	\ding{213}	&	214	&	\ding{214}	&	215	&	\ding{215} \\ \hline
216	&	\ding{216}	&	217	&	\ding{217}	&	218	&	\ding{218}	&	219	&	\ding{219}	&	220	&	\ding{220}	&	221	&	\ding{221}	&	222	&	\ding{222}	&	223	&	\ding{223} \\ \hline
224	&	\ding{224}	&	225	&	\ding{225}	&	226	&	\ding{226}	&	227	&	\ding{227}	&	228	&	\ding{228}	&	229	&	\ding{229}	&	230	&	\ding{230}	&	231	&	\ding{231} \\ \hline
232	&	\ding{232}	&	233	&	\ding{233}	&	234	&	\ding{234}	&	235	&	\ding{235}	&	236	&	\ding{236}	&	237	&	\ding{237}	&	238	&	\ding{238}	&	239	&	\ding{239} \\ \hline
	&		&	241	&	\ding{241}	&	242	&	\ding{242}	&	243	&	\ding{243}	&	244	&	\ding{244}	&	245	&	\ding{245}	&	246	&	\ding{246}	&	247	&	\ding{247} \\ \hline
248	&	\ding{248}	&	249	&	\ding{249}	&	250	&	\ding{250}	&	251	&	\ding{251}	&	252	&	\ding{252}	&	253	&	\ding{253}	&	254	&	\ding{254}	&		&	 \\ \hline
  \end{tabular}
  \caption{Os caracteres na fonte PostScript Zapf Dingbats}
\end{table}

\section{Catálogo de fontes}

% texdoc fontname

% COLOCAR O LINK DE CADA FONTE.

Vejamos alguns exemplos do site \href{http://www.tug.dk/FontCatalogue/}{The \LaTeX\ Font Catalogue}.

O site agrupa as fontes por categorias. Veremos apenas algumas fontes.

Para cada exemplo, o site mostra como usar a fonte no documento todo, geralmente carregando um pacote, mas no arquivo \href{}{font\_catalogue.tex} veremos como usar a \textbf{família} da fonte apenas num texto selecionado.





Baixe \href{}{fonte\_catalogue.tex}.


\section{Instalando novas fontes}


\subsection{Instalando fontes não free no Linux}

http://tex.stackexchange.com/a/22162/1940

Baixe install-getnonfreefonts
http://www.tug.org/fonts/getnonfreefonts/install-getnonfreefonts

% $sudo texlua install-getnonfreefonts

% ou

% $wget http://tug.org/fonts/getnonfreefonts/install-getnonfreefonts
% $sudo texlua install-getnonfreefonts

% Atualize
% sudo mktexlsr

% Falhou
% Testando
% tex testfont #ou pdftex testfont
% digite o nome da fonte

% Tente
% kpsewhich luximono.map



Instalando luximono

sudo getnonfreefonts-sys luximono

Veja a lista de fontes que podem ser instaladas.
getnonfreefonts -l

\subsection{Instalando mais algumas fontes no LaTeX}

Instalando Garamond

sudo mkdir /usr/local/texlive/2012/texmf-dist/fonts/afm/urw/garamond
sudo cp *.afm /usr/local/texlive/2012/texmf-dist/fonts/afm/urw/garamond
sudo mkdir /usr/local/texlive/2012/texmf-dist/fonts/type1/urw/garamond
sudo cp *.pfb /usr/local/texlive/2012/texmf-dist/fonts/type1/urw/garamond
sudo texhash /usr/local/texlive/2012/texmf-dist



\subsection{Instalando fontes no Linux}

Instalando fonte Linotype Zapfino Three no Linux

http://font.downloadatoz.com/font,11544,linotypezapfino-three.html

O modo mais simples é clicando no arquivo baixado LinotypeZapfino Three.ttf e ao abrir o visualizador de fontes clicar em Instalar

figura do visualizador de fontes

% + site: http://howtoubuntu.org/how-to-install-fonts-in-ubuntu/

OU via terminal

- Descompacte e crie uma subpasta em /usr/share/fonts/truetype/
mkdir /usr/share/fonts/truetype/linotypeZapfino

Copie o arquivo LinotypeZapfino Three.ttf para a pasta criada
sudo cp Downloads/LinotypeZapfino\ Three.ttf /usr/share/fonts/truetype/linotypeZapfino/


ou 

mkdir /usr/share/fonts/truetype/customfonts

e copie todas as fontes que deseja instalar para esta pasta

sudo cp *.ttf /usr/share/fonts/truetype/customfonts/

Depois atualize a lista de fontes
fc-cache -fv

Se quiser listar a fonte LinotypeZapfino instalada digite
fc-list | grep -i linotype


Instalando fonte xkcd
wget http://simonsoftware.se/other/xkcd.ttf

Instalando fonte xkcd
wget http://simonsoftware.se/other/xkcd.ttf

% 
% http://tex.stackexchange.com/questions/19898/getting-urw-garamond-and-the-license
% 
% instalar:
% covfonts
% malvern
% EMERALD
% vicent
% punk
% Segment Font
% vivaldi

% tentar instalar na força bruta
% instalar mais fontes ttf de outros lugares



\section{Ligaduras}

\href{http://pt.wikipedia.org/wiki/Ligadura_tipográfica}{Ligaduras} ocorrem quando duas ou mais letras são unidas em um único glifo. Para ignorar essas ocorrências use o comando \verb|\mbox{}|.

\verb|\Huge fi ff ffi fl|

{\Huge fi ff ffi fl}

\verb|f\mbox{}i f\mbox{}f f\mbox{}f\mbox{}i f\mbox{}l|

{\Huge f\mbox{}i f\mbox{}f f\mbox{}f\mbox{}i f\mbox{}l

filosofia

f\mbox{}ilosof\mbox{}ia}




% ver mais fontes decorativas
% fontes caligraficas para texto de capitulo
% iniciais bem desenhadas




% 
% 
% frase:
% \frase
% 
% 
% A vida só pode ser compreendida olhando-se para trás, mas só pode ser vivida olhando-se para frente. (Johnnie Walker)
% 
% 
% 
% - história de cada tópico (intro) em 'fntguide'
% 
% 
% 
% 
% \href{http://ctan.tche.br/macros/latex/required/psnfss/psnfss2e.pdf}{Using common PostScript fonts with LaTeX} ou texdoc psnfss2e
% -> tabela com as fontes mais comuns do LaTeX
% 
% colocar o link no final do post
% \href{http://pessoa.fct.unl.pt/p110371/nuno/category/typography/}{Tipografia (especificações técnicas de uma caixa de texto)}
% 
% 
% 
% \href{http://ctan.tche.br/info/fontsampler/sampler.pdf}{fontes disponíveis no TeXLive}
% \href{http://suppiya.files.wordpress.com/2008/02/latex_fonts.pdf}{LaTeX Fonts} de Ki-Joo Kim
% \href{http://linorg.usp.br/CTAN/info/symbols/comprehensive/symbols-a4.pdf}{symbols} de Scott Pakin
% \href{}{}
% 
% 
% % a nice font for the title
% \usepackage[sf]{vivaldi}
% 
% microtype
% 
% Referências
% 
% 
% texdoc classes
% 
% \fontencoding{T1}
% \fontfamily{garamond}
% \fontseries{m}
% \fontshape{it}
% \fontsize{12}{15}
% \selectfont
% 
% 
% 
% Outras fontes
% 
% ver as fontes do psnfss2e pág. 8, lista-las e fazer exemplos. - OK
% 
% - testar as fontes do sistema e ver quais são as fontes instaladas no ubuntu - OK
% 
% 
% texdoc mathalfa
% mathdesign
% 
% http://ctan.tche.br/help/Catalogue/entries/helvet.html
% http://www.tug.dk/FontCatalogue/helvetica/
% 
% http://www.tug.org/fontname/html/Standard-PostScript-fonts.html
% 
% \href{http://www.tug.dk/FontCatalogue/}{The \LaTeX{} Font Catalogue}
% % \href{http://www.tug.dk/FontCatalogue/helvetica/}{helvetica}
% \href{http://ctan.tche.br/help/Catalogue/entries/helvet.html}{helvet}
% 
% \href{http://www.tug.org/fontname/html/Standard-PostScript-fonts.html}{PostScript-fonts}

% LER ESTE SITE
% \href{http://tex.stackexchange.com/questions/56851/how-to-use-fonts-that-need-different-fontenc-together}{}
% sobre letrine
% 
% \href{}{}
% \href{http://linorg.usp.br/CTAN/macros/latex/doc/fntguide.pdf}{LaTeX2e font selection} ou texdoc fntguide
% \href{http://www.tug.org/pracjourn/2006-1/schmidt/schmidt.pdf}{Font selection in LATEX: The most frequently asked questions}
% \href{http://tug.org/fonts/}{Fonts and TeX}
% (1) \href{http://en.wikibooks.org/wiki/LaTeX/Text_Formatting}{Wikibooks fonts}
% \href{http://tex.stackexchange.com/questions/25249/how-do-i-use-a-particular-font-for-a-small-section-of-text-in-my-document/25251#25251}{Finding the font name (TeX.SX)}
% \href{http://www.forkosh.com/pstex/latexcommands.htm}{LaTeX Fonts Commands}
% \href{http://tex.stackexchange.com/a/25251/1940}{Restricting the scope of the selection}

% \href{http://www.tecepe.eng.br/blog/tex/latex/latex-codificacao-de-entrada-e-de-saida}{LaTeX - Codificação de entrada e de saída}

%\href{}{mathmode}
% \href{}{}

% VER http://ctan.tche.br/help/Catalogue/bytopic.html#fontsgraph

%% SER SE CONSEGUE COLOCAR CONCRETE E IWONA NOVAMENTE COM UMA DECLARAÇÃO PARTICULAR PARA AS EXPRESSÕES MATEMÁTICAS.

%% REVER A FONTE PADRÃO DESTE DOCUMENTO. TALVEZ TROCAR POR OUTRA. VER AS DA MEMOIR.

%% VER rodos os renewcommand em fntguide.

%% CRIAR UM fonte_modelo.tex com todos os comandos.

%% http://tex.stackexchange.com/a/74881/1940

%% http://packages.ubuntu.com/hardy/texlive-fonts-extra

%% http://www.fontsquirrel.com/fonts/list/50/800

%% http://en.kioskea.net/faq/1140-installing-truetype-fonts-under-ubuntu

%% http://font.downloadatoz.com/download,11563,linotypezapfino-two-for-linux.html

%% http://ilovetypography.com/

%% http://linorg.usp.br/CTAN/macros/latex/contrib/fontspec/fontspec.pdf

%% http://www.pctex.com/mtpro2.html

%% http://nitens.org/taraborelli/latex

%% falar sobre ligaduras

% Palavras-chave: Tipografia, Fontes, trabalhando com fontes no LaTeX, editando fontes no LaTeX.

\end{document}
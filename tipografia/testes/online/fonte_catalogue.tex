\documentclass[a4paper]{article}
\usepackage[utf8]{inputenc}
\usepackage[B1,T1]{fontenc}
\usepackage[brazil]{babel}
% \usepackage{parskip}
\usepackage{hyperref}
\hypersetup{pdfview = FitBH,pdfstartview = FitBH}
\usepackage{lipsum,xcolor}
\usepackage{lxfonts,ascii,aurical,auncial,bbold,lettrine,duerer}
\usepackage[margin=3cm]{geometry}

\renewcommand{\LettrineFontHook}{\initfamily}

%*************
\DeclareMathVersion{arev}

\SetSymbolFont{operators}    {arev}{OT1}{zavm}{m}{n}
\SetSymbolFont{letters}      {arev}{OML}{zavm}{m}{it}
\SetSymbolFont{symbols}      {arev}{OMS}{zavm}{m}{n}
\SetSymbolFont{largesymbols} {arev}{OMX}{mdbch}{m}{n}
\SetMathAlphabet{\mathnormal}{arev}{OML}{zavm}{m}{it}
\SetMathAlphabet{\mathit}    {arev}{OML}{zavm}{m}{it}
\SetMathAlphabet{\mathrm}    {arev}{OT1}{zavm}{m}{n}
\SetMathAlphabet{\mathsf}    {arev}{OML}{zavm}{m}{it}
\SetMathAlphabet{\mathbf}    {arev}{OT1}{zavm}{b}{n}
\SetMathAlphabet{\mathtt}    {arev}{T1} {fvm} {m}{n}
%*************
\DeclareMathVersion{kurier}
\DeclareMathVersion{kurierbold}

\SetSymbolFont{operators}    {kurier}{OT1}{kurier}{m}{n}
\SetSymbolFont{letters}      {kurier}{OML}{kurier} {m}{it}
\SetSymbolFont{symbols}      {kurier}{OMS}{kurier}{m}{n}
\SetSymbolFont{largesymbols} {kurier}{OMX}{kurier}{m}{n}
\SetSymbolFont{operators}    {kurierbold}{OT1}{kurierbold} {b}{n}
\SetSymbolFont{letters}      {kurierbold}{OML}{kurierbold} {b}{it}
\SetSymbolFont{symbols}      {kurierbold}{OMS}{kurierbold}{b}{n}
\SetSymbolFont{largesymbols} {kurierbold}{OMX}{kurierbold}{b}{n}
\SetMathAlphabet{\mathbf}    {kurier}{OT1}{kurier}{bx}{n}
\SetMathAlphabet{\mathsf}    {kurier}{OT1}{kurier}{m}{n}
\SetMathAlphabet{\mathit}    {kurier}{OT1}{kurier}{m}{it}
\SetMathAlphabet{\mathtt}    {kurier}{OT1}{kurier}{m}{n}
\SetMathAlphabet\mathsf      {kurierbold}{OT1}{kurierbold}{bx}{n}
\SetMathAlphabet\mathit      {kurierbold}{OT1}{kurierbold}{bx}{it}
%*************

% \newcommand{\frase}{Tem Coelhos que fogem de Abelhas Brigando por Jabuti porque faz chover uva e xadrez}

\newcommand{\frase}{The quick brown fox jumps over the sleazy dog}

\newcommand{\tipo}{TIPOGRAFIA}

% \newcommand{\funcaoint}{
%   f(t) = \frac{T}{2\pi} \int{\frac{1}{\sin\frac{\omega}{t}}}dt
% }

\newcommand{\funcaoint}{
  \mathbf{B}(P)=\frac{\mu_0}{4\pi}\int\frac{\mathbf{I}\times\hat{r}'}{r'^2}dl = \frac{\mu_0}{4\pi}\,I\!\int\frac{dl\times\hat{r}'}{r'^2}
}

% Fonte padrao Computer Modern Roman
\renewcommand\familydefault{cmr}
% Fonte padrao para typewriter
\renewcommand*\ttdefault{cmtt}

% familia
\newcommand{\fonte}[2]{
  {\fontfamily{#1}\selectfont #2}
}

\parindent=0pt
\pagestyle{empty}
%*******************************************************************
\begin{document}

Fontes descritas em \href{http://www.tug.dk/FontCatalogue/}{The \LaTeX\ Font Catalogue}.

\section*{Fontes com suporte matemático}

Se for usar no \textbf{documento todo} carregue \verb|\usepackage{pacote}|.

Se for usar apenas num \textbf{texto selecionado} use a família \texttt{familia}.

Usaremos a frase \href{http://pt.wikipedia.org/wiki/The_quick_brown_fox_jumps_over_the_lazy_dog}{The quick brown fox jumps over the sleazy dog} por ser um \href{http://pt.wikipedia.org/wiki/Pangrama}{pangrama} utilizado nos exemplos do site para efeito de comparação.

Para todos os exemplos a seguir usaremos os seguintes comandos (considerando apenas um texto selecionado):

\begin{verbatim}
\newcommand{\frase}{The quick brown fox jumps over the sleazy dog}
\newcommand{\fonte}[2]{{\fontfamily{#1}\selectfont #2}}
\fonte{familia}{\frase}
\emph{\fonte{familia}{\frase}}
\textbf{\fonte{familia}{\frase}}
{\mathversion{familia}$\funcaoint$}
\end{verbatim}

\textbf{Obs}: Usaremos \verb|\usepackage[T1]{fontenc}| por padrão. Mas existem fontes com saída \verb|[OT1]| ou \verb|[B1]|. Só que os dois últimos geralmente dão conflito, é recomendável que use somente um deles.

\subsection*{\href{http://www.tug.dk/FontCatalogue/arev/}{Arev}}

\begin{verbatim}
\usepackage{arev}
\usepackage[T1]{fontenc}
\end{verbatim}

Família: \texttt{fav}

\verb|\fonte{fav}{\frase}|

\

Para configurar a fonte \texttt{arev} para o \textit{modo matemático} é necessário digitar o seguinte código no preâmbulo (isto não será necessário caso você carregue o pacote):

\begin{verbatim}
\DeclareMathVersion{arev}

\SetSymbolFont{operators}    {arev}{OT1}{zavm}{m}{n}
\SetSymbolFont{letters}      {arev}{OML}{zavm}{m}{it}
\SetSymbolFont{symbols}      {arev}{OMS}{zavm}{m}{n}
\SetSymbolFont{largesymbols} {arev}{OMX}{mdbch}{m}{n}
\SetMathAlphabet{\mathnormal}{arev}{OML}{zavm}{m}{it}
\SetMathAlphabet{\mathit}    {arev}{OML}{zavm}{m}{it}
\SetMathAlphabet{\mathrm}    {arev}{OT1}{zavm}{m}{n}
\SetMathAlphabet{\mathsf}    {arev}{OML}{zavm}{m}{it}
\SetMathAlphabet{\mathbf}    {arev}{OT1}{zavm}{b}{n}
\SetMathAlphabet{\mathtt}    {arev}{T1} {fvm} {m}{n}
\end{verbatim}

E então digitar

\verb|{\mathversion{arev}$\funcaoint$}|

\

{\mathversion{arev}\[\funcaoint\]}

Veja \href{}{arev.tex}

\newpage 

\subsection*{\href{http://www.tug.dk/FontCatalogue/gfsartemisia/}{GFS Artemisia}}

\begin{verbatim}
\usepackage{gfsartemisia}
\usepackage[T1]{fontenc}
\end{verbatim} 

Família: \texttt{artemisia}

\fonte{artemisia}{\frase}

\hspace{-1ex}\emph{\fonte{artemisia}{\frase}}

\hspace{-1ex}\textsc{\fonte{artemisia}{\frase}}

\hspace{-1ex}\textbf{\fonte{artemisia}{\frase}}

O modo matemático de \textit{GFS Artemisia} é o padrão.\footnote{Neste caso a fonte não foi configurada, ou seja, a fonte padrão é \texttt{lxfonts} porque foi carregada no preâmbulo.}

\[
\funcaoint
\]

\subsection*{\href{http://www.tug.dk/FontCatalogue/kurierl/}{Kurier Light}}

\begin{verbatim}
\usepackage[light,math]{kurier}
\usepackage[T1]{fontenc}
\end{verbatim} 

Família: \texttt{kurierl}

\fonte{kurierl}{\frase}

\hspace{-1ex}\emph{\fonte{kurierl}{\frase}}

\hspace{-1ex}\textbf{\fonte{kurierl}{\frase}}

\begin{verbatim}
\DeclareMathVersion{kurier}
\DeclareMathVersion{kurierbold}

\SetSymbolFont{operators}    {kurier}{OT1}{kurier}{m}{n}
\SetSymbolFont{letters}      {kurier}{OML}{kurier} {m}{it}
\SetSymbolFont{symbols}      {kurier}{OMS}{kurier}{m}{n}
\SetSymbolFont{largesymbols} {kurier}{OMX}{kurier}{m}{n}
\SetSymbolFont{operators}    {kurierbold}{OT1}{kurierbold} {b}{n}
\SetSymbolFont{letters}      {kurierbold}{OML}{kurierbold} {b}{it}
\SetSymbolFont{symbols}      {kurierbold}{OMS}{kurierbold}{b}{n}
\SetSymbolFont{largesymbols} {kurierbold}{OMX}{kurierbold}{b}{n}
\SetMathAlphabet{\mathbf}{kurier}{OT1}{kurier}{bx}{n}
\SetMathAlphabet{\mathsf}{kurier}{OT1}{kurier}{m}{n}
\SetMathAlphabet{\mathit}{kurier}{OT1}{kurier}{m}{it}
\SetMathAlphabet{\mathtt}{kurier}{OT1}{kurier}{m}{n}
\SetMathAlphabet\mathsf{kurierbold}{OT1}{kurierbold}{bx}{n}
\SetMathAlphabet\mathit{kurierbold}{OT1}{kurierbold}{bx}{it}
\end{verbatim} 

{\mathversion{kurier}\[\funcaoint\]}

\newpage 

\subsection*{\href{http://www.tug.dk/FontCatalogue/lxfonts/}{LX Fonts}}

\begin{verbatim}
\usepackage{lxfonts}
\usepackage[T1]{fontenc}
\end{verbatim} 

Família: \texttt{llcmss}

\textbf{Obs}: É necessário carregar o pacote. Então se quiser que a \textit{Computer Modern Roman} continue como padrão no documento digite \verb|\renewcommand\familydefault{cmr}| no preâmbulo.

\fonte{llcmss}{\frase}

\hspace{-1ex}\emph{\fonte{llcmss}{\frase}}

\hspace{-1ex}\textbf{\fonte{llcmss}{\frase}}

\[
\funcaoint
\]

\

Veja um slide feito com \href{http://linorg.usp.br/CTAN/fonts/lxfonts/doc/fonts/lxfonts/LXfonts-demo.pdf}{lxfonts (slide)}.



\section*{Fontes com Serifa}

\subsection*{\href{http://www.tug.dk/FontCatalogue/antiqua/}{Antiqua}}

\begin{verbatim}
\usepackage{antiqua}
\usepackage[T1]{fontenc}
\end{verbatim} 

Família: \texttt{uaq}

\fonte{uaq}{\frase}

\fonte{uaq}{1 2 3 4 5 6 7 8 9 0}

% \subsection*{\href{http://www.tug.dk/FontCatalogue/covfonts/}{Covington}}
% 
% \verb|\usepackage{covfonts}|
% \usepackage[T1]{fontenc}
% 
% Família: \texttt{fkt}
% 
% \fonte{fkt}{\frase}
% 
% \hspace{-1ex}\emph{\fonte{fkt}{\frase}}
% 
% \hspace{-1ex}\textbf{\fonte{fkt}{\frase}}



\subsection*{\href{http://www.tug.dk/FontCatalogue/droidserif/}{Droid Serif}}

\begin{verbatim}
\usepackage[default]{droidserif}
\usepackage[T1]{fontenc}
\end{verbatim} 

Família: \texttt{fdr}

\fonte{fdr}{\frase}

\hspace{-1ex}\emph{\fonte{fdr}{\frase}}

\hspace{-1ex}\textbf{\fonte{fdr}{\frase}}

\fonte{fdr}{1 2 3 4 5 6 7 8 9 0}



\subsection*{\href{http://www.tug.dk/FontCatalogue/electrum/}{Electrum ADF}}

\begin{verbatim}
\usepackage[lf]{electrum}
\usepackage[T1]{fontenc}
\end{verbatim} 

Só funciona no documento todo. Veja o exemplo \href{}{fonte\_electrumadf.tex}.



\subsection*{\href{http://www.tug.dk/FontCatalogue/venturisold/}{Venturis ADF Old}}

\begin{verbatim}
\usepackage{yvo}
\usepackage[T1]{fontenc}
\end{verbatim} 

Família: \texttt{yvo}

\fonte{yvo}{\frase}

\hspace{-1ex}\emph{\fonte{yvo}{\frase}}

\hspace{-1ex}\textsc{\fonte{yvo}{\frase}}

\hspace{-1ex}\textbf{\fonte{yvo}{\frase}}

\fonte{yvo}{1 2 3 4 5 6 7 8 9 0}




\section*{Fontes sem Serifa}

\subsection*{\href{http://www.tug.dk/FontCatalogue/biolinum/}{Biolinum}}

A fonte \texttt{Biolinum} é a fonte padrão do pacote \textit{Libertine Legacy} que desde 2011 só é suportado pelos mecanismos \textit{xetex} e \textit{luatex}. Então no \LaTeX\ use

\begin{verbatim}
\usepackage{libertine}
\usepackage[T1]{fontenc}
\end{verbatim} 

Família: \texttt{fxb}

\fonte{fxb}{\frase}

\hspace{-1ex}\emph{\fonte{fxb}{\frase}}

\hspace{-1ex}\textbf{\fonte{fxb}{\frase}}

\fonte{fxb}{1 2 3 4 5 6 7 8 9 0}



\subsection*{\href{http://www.tug.dk/FontCatalogue/comfortaa/}{Comfortaa}}

\begin{verbatim}
\usepackage[default]{comfortaa}
\usepackage[T1]{fontenc}
\end{verbatim} 

Família: \texttt{fco}

\fonte{fco}{\frase}

\hspace{-1ex}\emph{\fonte{fco}{\frase}}

\hspace{-1ex}\textbf{\fonte{fco}{\frase}}

\fonte{fco}{1 2 3 4 5 6 7 8 9 0}



\subsection*{\href{http://www.tug.dk/FontCatalogue/lato/}{Lato}}

\begin{verbatim}
\usepackage[default]{lato}
\usepackage[T1]{fontenc}
\end{verbatim} 

Família: \texttt{fla}

\fonte{fla}{\frase}

\hspace{-1ex}\emph{\fonte{fla}{\frase}}

\hspace{-1ex}\textbf{\fonte{fla}{\frase}}

\fonte{fla}{1 2 3 4 5 6 7 8 9 0}

% \subsection*{\href{http://www.tug.dk/FontCatalogue/malvern/}{Malvern}}
% 
% Para usar esta fonte digite
% 
% \begin{verbatim}
% \input T1fmv.fd
% \renewcommand*\sfdefault{fmv}
% \renewcommand*\familydefault{\sfdefault} %% Somente se a fonte principal do documento for sem serifa.
% \usepackage[T1]{fontenc}
% \end{verbatim}
% 
% Família: \texttt{}
% 
% \fonte{fmv}{\frase}
% 
% \hspace{-1ex}\emph{\fonte{fmv}{\frase}}
% 
% \hspace{-1ex}\textbf{\fonte{fmv}{\frase}}
% 
% \fonte{fmv}{1 2 3 4 5 6 7 8 9 0}
% 


\subsection*{\href{http://www.tug.dk/FontCatalogue/tapir/}{Tapir}}

Para usar esta fonte digite no preâmbulo:

\newfont{\tap}{tap scaled 1200}

\verb|\newfont{\tap}{tap scaled 1200}|

Para usar digite \verb|{\tap \frase}|

{\tap \frase}

{\tap 1 2 3 4 5 6 7 8 9 0}

Não existe o modo itálico e nem negrito.



\subsection*{\href{http://www.tug.dk/FontCatalogue/grotesk/}{URW Grotesk}}

Para usar esta fonte digite no preâmbulo:

\renewcommand*\sfdefault{ugq}

\begin{verbatim}
\renewcommand*\sfdefault{ugq}
\usepackage[T1]{fontenc}
\end{verbatim} 

Família: \texttt{ugq}

\fonte{ugq}{\frase}

\hspace{-1ex}\emph{\fonte{ugq}{\frase}}

\hspace{-1ex}\textbf{\fonte{ugq}{\frase}}

\fonte{ugq}{1 2 3 4 5 6 7 8 9 0}



\subsection*{\href{http://www.tug.dk/FontCatalogue/universal/}{Universal}}

\begin{verbatim}
\usepackage[default]{uni}
\usepackage[OT1]{fontenc} %% Universal nao funciona com T1
\end{verbatim} 

Veja \href{}{fonte\_universal.tex}



\section*{Fontes máquina de escrever}

\subsection*{\href{http://www.tug.dk/FontCatalogue/ascii/}{Ascii}}

\begin{verbatim}
\usepackage{ascii}
\usepackage[T1]{fontenc}
\end{verbatim} 

\textbf{Obs}: É necessário carregar o pacote.

Para usar digite \verb|{\normalfont\asciifamily \frase}|.

{\normalfont\asciifamily \frase

1 2 3 4 5 6 7 8 9 0}

% \subsection*{\href{http://www.tug.dk/FontCatalogue/luximono/}{LuxiMono}}
% 
% \begin{verbatim}
% \usepackage{luximono}
% % \renewcommand*\familydefault{\ttdefault} %% Somente se a fonte principal do documento for maquina de escrever.
% \usepackage[T1]{fontenc}
% \end{verbatim}
% 
% Família: \texttt{}
% 
% \fonte{}{\frase}
% 
% \hspace{-1ex}\emph{\fonte{}{\frase}}
% 
% \hspace{-1ex}\textbf{\fonte{}{\frase}}
% 
% \fonte{}{1 2 3 4 5 6 7 8 9 0}



\subsection*{\href{http://www.tug.dk/FontCatalogue/txtt/}{TXTT}}

Para usar esta fonte digite no preâmbulo:

\begin{verbatim}
\renewcommand*\ttdefault{txtt}
\renewcommand*\familydefault{\ttdefault} %% Somente se a fonte principal do documento
  for maquina de escrever.
\usepackage[T1]{fontenc}
\end{verbatim}

Família: \texttt{txtt}

% \renewcommand*\ttdefault{txtt}

\fonte{txtt}{\frase}

\hspace{-1ex}\textsc{\fonte{txtt}{\frase}}

\hspace{-1ex}\textbf{\fonte{txtt}{\frase}}

\fonte{txtt}{1 2 3 4 5 6 7 8 9 0}


\subsection*{\href{http://www.tug.dk/FontCatalogue/tgcursor/}{\TeX\ Gyre Cursor}}

\begin{verbatim}
\usepackage{tgcursor}
\renewcommand*\familydefault{\ttdefault} %% Somente se a fonte principal do documento
  for maquina de escrever.
\usepackage[T1]{fontenc}
\end{verbatim} 

Família: \texttt{qcr}

\fonte{qcr}{\frase}

\hspace{-1ex}\emph{\fonte{qcr}{\frase}}

\hspace{-1ex}\textsc{\fonte{qcr}{\frase}}

\hspace{-1ex}\textbf{\fonte{qcr}{\frase}}

\fonte{qcr}{1 2 3 4 5 6 7 8 9 0}



\newpage 

\section*{Fontes caligráficas e manuscritas}

\subsection*{\href{http://www.tug.dk/FontCatalogue/aurical/}{Auriocus Kalligraphicus}}

É necessário carregar o pacote.

\begin{verbatim}
\usepackage{aurical}
\usepackage[T1]{fontenc}
\end{verbatim} 

Para usar num texto selecionado digite

\verb|{\Fontauri{\frase}}|  \hfill {\Fontauri{\frase}}

\verb|{\Fontauri\bfseries{\frase}}| \hfill {\Fontauri\bfseries{\frase}}

\verb|{\Fontlukas{\frase}}| \hfill {\Fontlukas{\frase}}

\verb|{\Fontamici{\frase}}| \hfill {\Fontamici{\frase}}

\verb|{\Fontauri{1 2 3 4 5 6 7 8 9 0}}| \hfill {\Fontauri{1 2 3 4 5 6 7 8 9 0}}



\subsection*{\href{http://www.tug.dk/FontCatalogue/frcursive/}{French Cursive}}

\begin{verbatim}
\usepackage[default]{frcursive}
\usepackage[T1]{fontenc}
\end{verbatim} 

Família: \texttt{frc}

\fonte{frc}{\frase}

\hspace{-1ex}\emph{\fonte{frc}{\frase}}

\hspace{-1ex}\textbf{\fonte{frc}{\frase}}

\fonte{frc}{1 2 3 4 5 6 7 8 9 0}



\subsection*{\href{http://www.tug.dk/FontCatalogue/janaskrivana/}{Jana Skrivana}}

É necessário carregar o pacote.

\begin{verbatim}
\usepackage{aurical}
\usepackage[T1]{fontenc}
\end{verbatim} 

Para usar num texto selecionado digite

\verb|{\Fontskrivan{\frase}}| 

{\Fontskrivan{\frase}}

{\Fontskrivan\bfseries{\frase}}

{\Fontskrivan{1 2 3 4 5 6 7 8 9 0}}


\subsection*{\href{http://www.tug.dk/FontCatalogue/tgchorus/}{\TeX\ Gyre Chorus}}

\begin{verbatim}
\usepackage{tgchorus}
\usepackage[T1]{fontenc}
\end{verbatim} 

Família: \texttt{qzc}

\fonte{qzc}{\frase}

\fonte{qzc}{1 2 3 4 5 6 7 8 9 0}

Nota: Esta fonte é idêntica a \textit{Zapf Chancery}.



% \subsection*{\href{http://www.tug.dk/FontCatalogue/teenspirit/}{Teen Spirit}}
% 
% \verb|\usepackage{emerald}|
% 
% Família: \texttt{}
% 
% \fonte{}{\frase}
% 
% \hspace{-1ex}\emph{\fonte{}{\frase}}
% 
% \hspace{-1ex}\textbf{\fonte{}{\frase}}
% 
% \fonte{}{1 2 3 4 5 6 7 8 9 0}
% 
% 
% 
% \subsection*{\href{http://www.tug.dk/FontCatalogue/vicentino/}{Vicentino}}
% 
% \verb|\usepackage{vicent}|
% 
% Família: \texttt{}
% 
% \fonte{}{\frase}
% 
% \hspace{-1ex}\emph{\fonte{}{\frase}}
% 
% \hspace{-1ex}\textbf{\fonte{}{\frase}}
% 
% \fonte{}{1 2 3 4 5 6 7 8 9 0}





\section*{Fontes Uncial}

\subsection*{\href{http://www.tug.dk/FontCatalogue/auncial/}{Artificial Uncial}}

Para usar esta fonte necessário carregar o pacote. Além disso, se você usar dois \texttt{fontenc} no mesmo documento coloque \texttt{B1} primeiro para evitar resultados indesejados nas demais fontes.

\begin{verbatim}
\usepackage{auncial}
\usepackage[B1,T1]{fontenc} %% auncial so funciona com B1
\end{verbatim}

Para usar digite \verb|{\normalfont\aunclfamily \frase}|.

{\normalfont\aunclfamily \frase

1 2 3 4 5 6 7 8 9 0}




\subsection*{\href{http://www.tug.dk/FontCatalogue/carolmin/}{Carolingan Miniscules}}

\begin{verbatim}
\usepackage{carolmin}
\usepackage[T1]{fontenc}
\end{verbatim} 

Família: \texttt{cmin}

\fonte{cmin}{\frase}

\hspace{-1ex}\textbf{\fonte{cmin}{\frase}}

\fonte{cmin}{1 2 3 4 5 6 7 8 9 0}




\subsection*{\href{http://www.tug.dk/FontCatalogue/eiad/}{Eiad}}

\begin{verbatim}
\newcommand*\eiad{\fontencoding{OT1}\fontfamily{eiad}\selectfont}
\usepackage[OT1]{fontenc}
\end{verbatim} 

Família: \texttt{eiad}

\verb|\fonte{eiad}{\frase}|

ou

\verb|{\eiad \frase}|

\

Veja \href{}{fonte\_eiad.tex}


\subsection*{\href{http://www.tug.dk/FontCatalogue/huncial/}{Half Uncial}}

\begin{verbatim}
\usepackage{huncial}
\usepackage[T1]{fontenc}
\end{verbatim} 

Para usar no documento todo digite \verb|\normalfont\hunclfamily| dentro de \verb|\begin{document}|.

Família: \texttt{huncl}

\fonte{huncl}{\frase}

\fonte{huncl}{1 2 3 4 5 6 7 8 9 0}


\subsection*{\href{http://www.tug.dk/FontCatalogue/inslrmin/}{Insular Minuscule}}

\begin{verbatim}
\usepackage{inslrmin}
\usepackage[T1]{fontenc}
\end{verbatim} 

Para usar no documento todo digite \verb|\normalfont\iminfamily| dentro de \verb|\begin{document}|.

Família: \texttt{imin}

\fonte{imin}{\frase}

\hspace{-1ex}\textbf{\fonte{imin}{\frase}}

\fonte{imin}{1 2 3 4 5 6 7 8 9 0}



\subsection*{\href{http://www.tug.dk/FontCatalogue/rustic/}{Roman Rustic}}

\begin{verbatim}
\usepackage{rustic}
\usepackage[T1]{fontenc}
\end{verbatim} 

Para usar no documento todo digite \verb|\normalfont\rustfamily| dentro de \verb|\begin{document}|.

Família: \texttt{rust}

\fonte{rust}{\frase}

\fonte{rust}{1 2 3 4 5 6 7 8 9 0}


\subsection*{\href{http://www.tug.dk/FontCatalogue/rotunda/}{Rotunda}}

\begin{verbatim}
\usepackage{rotunda}
\usepackage[T1]{fontenc}
\end{verbatim} 

Para usar no documento todo digite \verb|\normalfont\rtndfamily| dentro de \verb|\begin{document}|.

Família: \texttt{rtnd}

\fonte{rtnd}{\frase}

\fonte{rtnd}{1 2 3 4 5 6 7 8 9 0}




\section*{Fontes Blackletter}

Veja \href{http://www.tex.ac.uk/tex-archive/help/Catalogue/entries/yfonts.html}{yfonts} em \href{}{Outras fontes}.


\section*{Outras fontes}

% \subsection*{\href{http://www.tug.dk/FontCatalogue/apicturealfabet/}{A Picture Alphabet}}
% 
% \verb|\usepackage{emerald}|
% 
% Família: \texttt{}
% 
% \fonte{}{\frase}
% 
% \hspace{-1ex}\emph{\fonte{}{\frase}}
% 
% \hspace{-1ex}\textbf{\fonte{}{\frase}}


\subsection*{\href{http://www.tug.dk/FontCatalogue/bbold/}{Bbold}}

\begin{verbatim}
\usepackage{bbold}
\usepackage[T1]{fontenc}
\end{verbatim} 

Para usar no documento todo digite \verb|\normalfont\bbfamily| dentro de \verb|\begin{document}|.

Para usar num texto selecionado digite

\verb|\textbb{\frase}|

\textbb{\frase

1 2 3 4 5 6 7 8 9 0}


% \subsection*{\href{http://www.tug.dk/FontCatalogue/decadence/}{Decadence}}
% 
% \verb|\usepackage{emerald}|
% 
% Família: \texttt{}
% 
% \fonte{}{\frase}
% 
% \hspace{-1ex}\emph{\fonte{}{\frase}}
% 
% \hspace{-1ex}\textbf{\fonte{}{\frase}}



% \subsection*{\href{http://www.tug.dk/FontCatalogue/movieola/}{Movieola}}
% 
% \verb|\usepackage{emerald}|
% 
% Família: \texttt{}
% 
% \fonte{}{\frase}
% 
% \hspace{-1ex}\emph{\fonte{}{\frase}}
% 
% \hspace{-1ex}\textbf{\fonte{}{\frase}}



% \subsection*{\href{http://www.tug.dk/FontCatalogue/pookie/}{Pookie}}
% 
% \verb|\usepackage{emerald}|
% 
% Família: \texttt{}
% 
% \fonte{}{\frase}
% 
% \hspace{-1ex}\emph{\fonte{}{\frase}}
% 
% \hspace{-1ex}\textbf{\fonte{}{\frase}}


% INSTALAR
% \subsection*{\href{http://www.tug.dk/FontCatalogue/punk/}{Punk}}
% 
% \verb|\usepackage{punk}|
% 
% Família: \texttt{}
% 
% \fonte{}{\frase}
% 
% \hspace{-1ex}\emph{\fonte{}{\frase}}
% 
% \hspace{-1ex}\textbf{\fonte{}{\frase}}



% \subsection*{\href{http://www.tug.dk/FontCatalogue/dseg/}{Segment Font}}
% 
% Para usar esta fonte digite no preâmbulo:
% 
% % \newfont{\dviiseg}{d7seg scaled 1200}
% 
% \newfont{\deseg}{deseg scaled 1200}
% 
% \verb||
% 
% Para usar digite \verb|{\dviiseg \frase}|
% 
% % {\dviiseg \frase}
% 
% {\deseg \frase}
% 
% {\dviiseg 1 2 3 4 5 6 7 8 9 0}



% \subsection*{\href{http://www.tug.dk/FontCatalogue/webster/}{Webster}}
% 
% \verb|\usepackage{emerald}|
% 
% Família: \texttt{}
% 
% \fonte{}{\frase}
% 
% \hspace{-1ex}\emph{\fonte{}{\frase}}
% 
% \hspace{-1ex}\textbf{\fonte{}{\frase}}



\section*{Fontes em caixa alta}

\subsection*{\href{http://www.tug.dk/FontCatalogue/duerer/}{D\"urer}}

\begin{verbatim}
\usepackage{duerer}
\usepackage[T1]{fontenc}
\end{verbatim} 

Para usar no documento todo digite \verb|\normalfont\durmfamily| dentro de \verb|\begin{document}|.

Família: \texttt{cdr}

{\normalfont\durmfamily

THE QUICK BROWN FOX JUMPS OVER THE SLEAZY DOG}


\subsection*{\href{http://www.tug.dk/FontCatalogue/foekfont/}{FoekFont}}

\begin{verbatim}
\usepackage{foekfont}
\usepackage[T1]{fontenc}
\end{verbatim} 

Para usar no documento todo digite \verb|\normalfont\foekfamily| dentro de \verb|\begin{document}|.

Família: \texttt{foekfont}

\fonte{foekfont}{THE QUICK BROWN FOX JUMPS OVER THE SLEAZY DOG}


\subsection*{\href{http://www.tug.dk/FontCatalogue/rfsf/}{Ralph Smith's Formal Script Symbol Fonts}}

Para usar esta fonte digite no preâmbulo:

\newfont{\rsfsten}{rsfs10 scaled 1200}
\newfont{\rsfsseven}{rsfs10 scaled 1200}
\newfont{\rsfsfive}{rsfs10 scaled 1200}

\begin{verbatim}
\newfont{\rsfsten}{rsfs10 scaled 1200}
\newfont{\rsfsseven}{rsfs10 scaled 1200}
\newfont{\rsfsfive}{rsfs10 scaled 1200}
\end{verbatim} 

Para usar digite \verb|{\rsfsten TEXTO}| ou \verb|{\rsfsseven TEXTO}| ou \verb|{\rsfsfive TEXTO}|.

{\rsfsten THE QUICK BROWN FOX JUMPS OVER THE SLEAZY DOG}




\section*{Iniciais decorativas}

\subsection*{\href{http://www.tug.dk/FontCatalogue/acorn/}{Acorn Initials}}

Para usar esta fonte digite no preâmbulo:

\input Acorn.fd
\newcommand*\initfamily{\usefont{U}{Acorn}{xl}{n}}

\begin{verbatim}
\input Acorn.fd
\newcommand*\initfamily{\usefont{U}{Acorn}{xl}{n}}
\end{verbatim} 

Para usar digite \verb|{\initfamily TIPOGRAFIA}|

{\initfamily\Huge \tipo}

{\initfamily\Huge V}eja um exemplo numa frase grande. \lipsum[1]

\

O exemplo fica melhor aplicado se usado da seguinte forma:

\begin{verbatim}
\usepackage{lettrine}
\renewcommand{\LettrineFontHook}{\initfamily}
\end{verbatim}

Para usar digite

\begin{verbatim}
\lettrine[lines=4]{V}{eja} um exemplo numa frase grande. \lipsum[1]
\end{verbatim}

\noindent onde \verb|lines=4| é o número de linhas que você quer que a letra ocupe. Experimente também usar uma letra colorida, para isso carregue o pacote \texttt{xcolor}.

\begin{verbatim}
\lettrine[lines=4]{\textcolor{blue}{V}}{eja}           
\end{verbatim}

\lettrine[lines=4]{\textcolor{blue}{V}}{eja} um exemplo numa frase grande. \lipsum[1]


\subsection*{\href{http://www.tug.dk/FontCatalogue/annston/}{Ann Stone}}

Para usar esta fonte digite no preâmbulo:

\input AnnSton.fd
\renewcommand*\initfamily{\usefont{U}{AnnSton}{xl}{n}}

\begin{verbatim}
\input AnnSton.fd
\newcommand*\initfamily{\usefont{U}{AnnSton}{xl}{n}}
\end{verbatim}

\textbf{Obs}: Caso você tenha dois \verb|\newcommand*\initfamily{}| no mesmo documento digite o segundo como

\verb|\renewcommand*\initfamily{}|, ou use outro nome, exemplo \verb|\newcommand*\initfamilyA{}|.

Para usar digite \verb|{\initfamily TIPOGRAFIA}|

{\initfamily\Huge \tipo}

Veja o exemplo com \texttt{lettrine}.

\begin{verbatim}
\lettrine[lines=4]{V}{eja} um exemplo numa frase grande. \lipsum[1]
\end{verbatim}

\lettrine[lines=4]{V}{eja} um exemplo numa frase grande. \lipsum[1]


\subsection*{\href{http://www.tug.dk/FontCatalogue/artnouvc/}{Art Nouveau Caps}}

Para usar esta fonte digite no preâmbulo:

\input ArtNouvc.fd
\renewcommand*\initfamily{\usefont{U}{ArtNouvc}{xl}{n}}

\begin{verbatim}
\input ArtNouvc.fd
\newcommand*\initfamily{\usefont{U}{ArtNouvc}{xl}{n}}
\end{verbatim}

Para usar digite \verb|{\initfamily TIPOGRAFIA}|

{\initfamily\Huge \tipo}

Veja o exemplo com \texttt{lettrine}.

\begin{verbatim}
\lettrine[lines=4]{V}{eja} um exemplo numa frase grande. \lipsum[1]
\end{verbatim}

\lettrine[lines=4]{V}{eja} um exemplo numa frase grande. \lipsum[1]

\newpage 

\subsection*{\href{http://www.tug.dk/FontCatalogue/artnouv/}{Art Nouveau Initialen}}

Para usar esta fonte digite no preâmbulo:

\input ArtNouv.fd
\renewcommand*\initfamily{\usefont{U}{ArtNouv}{xl}{n}}

\begin{verbatim}
\input ArtNouv.fd
\newcommand*\initfamily{\usefont{U}{ArtNouv}{xl}{n}}
\end{verbatim}

Para usar digite \verb|{\initfamily TIPOGRAFIA}|

{\initfamily\Huge \tipo}

\

A aplicação dos exemplos a seguir são parecidos com os anteriores.

Para usar digite \verb|{\initfamily TIPOGRAFIA}|

\subsection*{\href{http://www.tug.dk/FontCatalogue/baroqueinitials/}{Baroque}}

Veja \href{http://www.tug.dk/FontCatalogue/baroqueinitials/}{Baroque} em \href{}{Outras fontes}.


\subsection*{\href{http://www.tug.dk/FontCatalogue/eileenbl/}{Eileen Caps Black}}

\input EileenBl.fd
\renewcommand*\initfamily{\usefont{U}{EileenBl}{xl}{n}}

\begin{verbatim}
\input EileenBl.fd
\newcommand*\initfamily{\usefont{U}{EileenBl}{xl}{n}}
{\initfamily TIPOGRAFIA}
\end{verbatim}

{\initfamily\Huge \tipo}


\subsection*{\href{http://www.tug.dk/FontCatalogue/elzevier/}{Elzevier Caps Regular}}

\input Elzevier.fd
\renewcommand*\initfamily{\usefont{U}{Elzevier}{xl}{n}}

\begin{verbatim}
\input Elzevier.fd
\newcommand*\initfamily{\usefont{U}{Elzevier}{xl}{n}}
{\initfamily TIPOGRAFIA}
\end{verbatim}

{\initfamily\Huge \tipo}


\subsection*{\href{http://www.tug.dk/FontCatalogue/konanur/}{Konanur Kaps}}

\input Konanur.fd
\renewcommand*\initfamily{\usefont{U}{Konanur}{xl}{n}}

\begin{verbatim}
\input Konanur.fd
\newcommand*\initfamily{\usefont{U}{Konanur}{xl}{n}}
{\initfamily TIPOGRAFIA}
\end{verbatim}

{\initfamily\Huge \tipo}



\subsection*{\href{http://www.tug.dk/FontCatalogue/rothdn/}{Rothenburg Decorative}}

\input Rothdn.fd
\renewcommand*\initfamily{\usefont{U}{Rothdn}{xl}{n}}

\begin{verbatim}
\input Rothdn.fd
\newcommand*\initfamily{\usefont{U}{Rothdn}{xl}{n}}
{\initfamily TIPOGRAFIA}
\end{verbatim}

{\initfamily\Huge \tipo}




\subsection*{\href{http://www.tug.dk/FontCatalogue/royalin/}{Royal Initialen}}

\input RoyalIn.fd
\renewcommand*\initfamily{\usefont{U}{RoyalIn}{xl}{n}}

\begin{verbatim}
\input RoyalIn.fd
\newcommand*\initfamily{\usefont{U}{RoyalIn}{xl}{n}}
{\initfamily TIPOGRAFIA}
\end{verbatim}

{\initfamily\Huge \tipo}




\subsection*{\href{http://www.tug.dk/FontCatalogue/sanremo/}{San Remo}}

\input Sanremo.fd
\renewcommand*\initfamily{\usefont{U}{Sanremo}{xl}{n}}

\begin{verbatim}
\input Sanremo.fd
\newcommand*\initfamily{\usefont{U}{Sanremo}{xl}{n}}
{\initfamily TIPOGRAFIA}
\end{verbatim}

{\initfamily\Huge \tipo}



\subsection*{\href{http://www.tug.dk/FontCatalogue/starburst/}{Starburst Regular}}

\input Starburst.fd
\renewcommand*\initfamily{\usefont{U}{Starburst}{xl}{n}}

\begin{verbatim}
\input Starburst.fd
\newcommand*\initfamily{\usefont{U}{Starburst}{xl}{n}}
{\initfamily TIPOGRAFIA}
\end{verbatim}

{\initfamily\Huge \tipo}



\subsection*{\href{http://www.tug.dk/FontCatalogue/typocaps/}{Typographer Caps}}

\input Typocaps.fd
\renewcommand*\initfamily{\usefont{U}{Typocaps}{xl}{n}}

\begin{verbatim}
\input Typocaps.fd
\newcommand*\initfamily{\usefont{U}{Typocaps}{xl}{n}}
{\initfamily TIPOGRAFIA}
\end{verbatim}

{\initfamily\Huge \tipo}




\subsection*{\href{http://www.tug.dk/FontCatalogue/zallman/}{Zallman Caps}}

\input Zallman.fd
\renewcommand*\initfamily{\usefont{U}{Zallman}{xl}{n}}

\begin{verbatim}
\input Zallman.fd
\newcommand*\initfamily{\usefont{U}{Zallman}{xl}{n}}
{\initfamily TIPOGRAFIA}
\end{verbatim}

{\initfamily\Huge \tipo}

\end{document}
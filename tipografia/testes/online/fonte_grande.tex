\documentclass[a4paper]{article}
\usepackage[utf8]{inputenc}
\usepackage[T1]{fontenc}
\usepackage[brazil]{babel}
\usepackage{fix-cm} %para texto grande
\usepackage{exscale} %para matematica
\usepackage{graphicx}
\usepackage[paperwidth=40cm,margin=2cm]{geometry}

\newcommand{\frase}{Tem Coelhos que fogem de Abelhas Brigando por Jabuti porque faz chover uva e xadrez}

\pagestyle{empty}

\begin{document}

\section*{Textos}

{\fontsize{100}{120}\selectfont Texto grande}

{\fontsize{200}{120}\selectfont Texto grande}

\newpage 

\section*{Expressões matemáticas}

\subsection*{Normal size}

\[
  \int_0^1 {\frac{\sqrt x}{2}dx}
\]

\subsection*{Huge}

O pacote \texttt{exscale} corrige o tamanho de todos os itens da expressão matemática.

{\Huge
\[
  \int_0^1 {\frac{\sqrt x}{2}dx} 
\]
}

\subsection*{Qualquer tamanho}

O pacote \texttt{graphicx} permite uma escala na expressão matemática em qualquer tamanho. Use o comando \verb|\scalebox{ }|.

\verb|\scalebox{0.5}|

\[
  \scalebox{0.5}{
    $\displaystyle\int_0^1 {\frac{\sqrt x}{2}dx}$
  }
\]

\verb|\scalebox{1}|

\[
  \scalebox{1}{
    $\displaystyle\int_0^1 {\frac{\sqrt x}{2}dx}$
  }
\]

\verb|\scalebox{10}|

\[
  \scalebox{10}{
    $\displaystyle\int_0^1 {\frac{\sqrt x}{2}dx}$
  }
\]

\end{document}